
% Basics und Codierung
% ===========================================================
\usepackage{wwustyle2}
\usepackage[ngerman]{babel}
\usepackage[utf8]{inputenc}
\usepackage[T1]{fontenc}
\usepackage[german=quotes]{csquotes}
\usepackage{scrtime}
\usepackage{etex}
\usepackage{shellesc}
\usepackage{wrapfig}
% ===========================================================


% Fonts und Typographie
% ===========================================================
\usepackage{sourcecodepro}
\usepackage[default]{sourcesanspro}
\usepackage{nimbusmononarrow}
\usepackage{ellipsis}
\newcommand{\bet}[1]{\textbf{\color{maincolor}#1}}
\newcommand{\minor}[1]{\textcolor{black!50}{#1}}
\newcommand{\minoritem}{\item[{\footnotesize \textcolor{black!50}{$\blacktriangleright$}}]}
\newcommand{\code}[1]{\texttt{#1}}
\usepackage{xspace}
\makeatletter 
\xspaceaddexceptions{\grqq \grq \csq@qclose@i \} } 
\makeatother
\usefonttheme[onlymath]{serif}
\usepackage{multicol}
\usepackage{multirow}
% ===========================================================

% Farben	
% ===========================================================
	\usepackage{xcolor}
	\definecolor{fbblau}{HTML}{3078AB}
	\definecolor{mediumgray}{gray}{.65}
	\definecolor{blackberry}{rgb}{0.53, 0.0, 0.25}
% ===========================================================

	
% Mathe-Pakete
% ===========================================================
	\usepackage{mathtools}
	\usepackage{amssymb}
	\usepackage[bigdelims]{newtxmath}

	% Abkürzungen
% ===========================================================
	\newcommand{\BB}{\mathbb{B}}
	\newcommand{\CC}{\mathbb{C}}
	\newcommand{\EE}{\mathbb{E}}
	\newcommand{\FF}{\mathbb{F}}
	\newcommand{\HH}{\mathcal{H}}
	\newcommand{\KK}{\mathbb{K}}
	\newcommand{\LL}{\mathbb{L}}
	\newcommand{\NN}{\mathbb{N}}
	\newcommand{\QQ}{\mathbb{Q}}
	\newcommand{\RR}{\mathbb{R}}
	\newcommand{\ZZ}{\mathbb{Z}}
	\newcommand{\oh}{\mathcal{O}}				% Landau-O
	\newcommand{\ind}{1\hspace{-0,8ex}1} 		% Indikatorfunktion (Doppeleins)
	\newcommand{\bewrueck}{\enquote{$\Leftarrow$}:$\:$} 	% Beweis Rückrichtung
	\newcommand{\bewhin}{\enquote{$\Rightarrow$}:$\:$}		% Beweis Hinrichtung
	\newcommand{\ol}[1]{\overline{#1}}
	\newcommand{\wt}[1]{\widetilde{#1}}
	\newcommand{\wh}[1]{\widehat{#1}}
% ===========================================================

% Operatoren
% ===========================================================
	\DeclareMathOperator{\id}{id} 				% Identität
	\DeclareMathOperator{\im}{im} 				% image
	\DeclareMathOperator{\pot}{\mathcal{P}}		% Potenzmenge
	\DeclareMathOperator{\sgn}{sgn} 			% Signum
	\DeclareMathOperator{\Sym}{Sym} 			% Symmetrische Gruppe
% ===========================================================

% Klammerungen und ähnliches
% ===========================================================
	\DeclarePairedDelimiter{\absolut}{\lvert}{\rvert}		% Betrag
	\DeclarePairedDelimiter{\ceiling}{\lceil}{\rceil}		% aufrunden
	\DeclarePairedDelimiter{\Floor}{\lfloor}{\rfloor}		% aufrunden
	\DeclarePairedDelimiter{\Norm}{\lVert}{\rVert}			% Norm
	\DeclarePairedDelimiter{\sprod}{\langle}{\rangle}		% spitze Klammern
	\DeclarePairedDelimiter{\enbrace}{(}{)}					% runde Klammern
	\DeclarePairedDelimiter{\benbrace}{\lbrack}{\rbrack}	% eckige Klammern
	\DeclarePairedDelimiter{\penbrace}{\{}{\}}				% geschweifte Klammern
	\newcommand{\Underbrace}[2]{{\underbrace{#1}_{#2}}} 	% bessere Unterklammerungen
	% Kurzschreibweisen für Faule und Code-Vervollständigung
	\newcommand{\abs}[1]{\absolut*{#1}}
	\newcommand{\ceil}[1]{\ceiling*{#1}}
	\newcommand{\flo}[1]{\Floor*{#1}}
	\newcommand{\no}[1]{\Norm*{#1}}
	\newcommand{\sk}[1]{\sprod*{#1}}
	\newcommand{\enb}[1]{\enbrace*{#1}}
	\newcommand{\penb}[1]{\penbrace*{#1}}
	\newcommand{\benb}[1]{\benbrace*{#1}}
% ===========================================================

	\usepackage{wasysym}
	
\newcommand{\zerodisplayskips}{%
%\setlength{\abovedisplayskip}{0pt}%
%\setlength{\belowdisplayskip}{0pt}%
%\setlength{\abovedisplayshortskip}{0pt}%
%\setlength{\belowdisplayshortskip}{0pt}
%\setlength{\multicolsep}{0pt}
}
\appto{\normalsize}{\zerodisplayskips}
\appto{\small}{\zerodisplayskips}
\appto{\footnotesize}{\zerodisplayskips}
% ===========================================================

% TikZ
% ===========================================================
	\usepackage{tikz}
	\usepackage{tikz-cd}					% kommutative Diagramme
	\usetikzlibrary{arrows.meta}			% mehr Pfeile!
	\usetikzlibrary{shadows}
	\usetikzlibrary{calc}
	\tikzset{>=Latex}						% Standard-Pfeilspitze
	\usetikzlibrary{automata,positioning}
	\usetikzlibrary{matrix}
% ===========================================================

% minted
% ===========================================================
\usepackage{minted}
\setminted{%
	style=default,
	fontsize=\footnotesize,
	breaklines,
	breakanywhere=false,
	breakbytoken=false,
	breakbytokenanywhere=false,
	breakafter={.,},
	autogobble,
	numbers=left,
	numbersep=3mm,
	tabsize=4,
	frame=lines
}
\setmintedinline{%
	fontsize=\normalsize,
	numbers=none,
	numbersep=12pt,
	tabsize=4,
	%bgcolor=gray!15,
}
% ===========================================================