%---------------------------------------------------------------------------
%
%                          Vorlage der Arbeitsgruppe
%             Computer Vision and Pattern Recognition Group (CVPR)
%                           der Universität Münster
%                         http://cvpr.uni-muenster.de
%
%---------------------------------------------------------------------------
% Geeignet für:
%  - Seminararbeiten
%  - Bachelorarbeiten
%  - Masterarbeiten
%---------------------------------------------------------------------------
% Autoren:
%  - Daniel Tenbrinck
%  - Fabian Gigengack
%  - Michael Schmeing
%  - Lucas Franek
%  - Andreas Nienkötter
%---------------------------------------------------------------------------
% Version:
%  - 1.0.3 (05.10.2016)
%	 - Ersetzung von veralteten Befehlen durch Aktuelle
%	 - Einige ausführlichere Beispiele
%    - Einführung von listings
%    - Aktuelle Version der Eidesstattlichen Erklärung
%  - 1.0.2 (09.09.2011)
%    - Titelblatt um Matrikelnummer und Studiengang ergänzt
%  - 1.0.1 (05.07.2011)
%---------------------------------------------------------------------------
% 
% "THE BEER-WARE LICENSE" (Revision 42):
% The above mentioned authors wrote this file. As long as you retain this
% notice you can do whatever you want with this stuff. If we meet some day,
% and you think this stuff is worth it, you can buy us a beer in return.
%  --------------------------------------------------------------------------

\documentclass[a4paper, twoside, 12pt, ngerman]{scrbook} % Layout-Einstellungen für das Dokument

\usepackage[utf8]{inputenc} % UTF-8 Codierung
\usepackage[ngerman]{babel} % Deutsche Beschriftung

\usepackage{graphicx} % Um Bilder einzufügen
\usepackage{subfigure} % Um mehrere Bilder in eine figure einzufügen
\usepackage{amssymb, amsmath} % Für Mengensymbole und über Gleichheitszeichen schreiben
\usepackage{verbatim} % Um Quellcode in das Dokument einzufügen.
\usepackage{xcolor} % Für Farben
\usepackage[linkbordercolor=blue]{hyperref} % Für Links im Dokument
\usepackage{algorithmic} % Für Pseudo-Code
\usepackage{algorithm} % Wrapper für Pseudo-Code
\usepackage[font={small}, labelfont=bf]{caption} % kleine Bildunterschriften
\usepackage{geometry} % Für Feinanpassungen des Layouts

\usepackage{listings} % Für Code-Listings
\renewcommand{\lstlistingname}{Quelltext} %Ändert die Überschrift von Listing nach Quelltext

% Einstellungen für Abstand an den Rändern
\geometry{a4paper,left=35mm,right=35mm,top=20mm,bottom=20mm, includeheadfoot}

\begin{document}
\pagenumbering{roman}

% Titelblatt
\begin{titlepage}
\begin{centering}
\vspace*{\fill}
\includegraphics[width=12cm]{./img/wwu-logo-neu.pdf}

\vspace{2cm} 

{\LARGE
	\textbf{Titel der Arbeit\\
			(mehrzeilig auch möglich)}\\[1.2cm]
}

{\large
	Bachelorarbeit oder Masterarbeit oder Seminararbeit\\[2cm]
}

{\large
	vorgelegt von:
}

{ \Large
	\textbf{Vorname Nachname}\\[1cm]
}

{\large
	Matrikelnummer: 123456\\[2mm]
}

{\large
	Studiengang: [Studiengang]\\[1cm]
}
    
{\large
	Thema gestellt von:
}

{\Large
	\textbf{Prof. Dr. Xiaoyi Jiang}\\[1cm]
}
                               
{\large
	Arbeit betreut durch:
}

{\Large
	\textbf{Vorname Nachname}\\[1cm]
}

{\large
Münster, \today
}
\vfill
\end{centering}
\end{titlepage}

% Inhaltsverzeichnis
\tableofcontents

\cleardoublepage
\pagenumbering{arabic}

% Die Hauptkapitel der Arbeit
\chapter{Einleitung und Motivation}
\label{ch:einleitung}

Das berechnen von multiplen \emph{Sequenzalignments} ist eine fundamentale Technik, die eine Vielzahl von Anwendungsgebieten in der Biologie, aber auch in anderen Disziplinen wie dem \emph{Natural Language Processing} hat. Wir werden im Laufe dieses Kapitels zunächst etwas darüber kennenlernen was \emph{Alignments} eigentlich sind  und wo ihre Hauptanwendungsgebiete liegen. Dann führen wir mit dem Algorithmus von Needleman-Wunsch einen Standardansatz für paarweise \emph{Alignments} ein. Am Beispiel dieses Algorithmus beschäftigen wir uns mit der Komplexität des \emph{multiple sequence Alignment}-Problems und stellen fest, dass es für Zuweisungen zwischen mehr als nur einigen wenigen Sequenzen nicht zielführend ist diese mathematisch exakt zu berechnen. Um diesem Problem Herr zu werden, lernen wir im Laufe der Bachelorarbeit zwei ausgefeilte Heuristiken für multiple \emph{Sequenzalignments} ein. Zunächst lernen wir DIALIGN kennen, einen Algorithmus, der anders als Needleman-Wunsch nicht auf der Basis von einzelnen Symbolen \emph{Alignments} konstruiert \citep{mdw96}. Stattdessen werden ganze Segmente als Bausteine der Zuweisungen benutzt. Der zweite Algorithmus ist der graphtheoretische Ansatz von \cite{cpm10}. Dieser basiert zwar auch auf DIALIGN, hat aber den Anspruch für Situationen bei denen die Heuristik von DIALIGN falsche Entscheidungen trifft, bessere Ergebnisse zu liefern. Exemplarisch wird danach ein wichtiger Schritt des Verfahrens programmatisch umgesetzt. Dabei wird mit der Programmiersprache C++ und der \enquote{Boost Graph Library} gearbeitet.

\section{Einführung und Anwendungsgebiete}

Ziel von \emph{Sequenzalignments} ist es für eine Menge von Zeichenketten aus einem endlichen Alphabet Zuordnungen zwischen den einzelnen Symbolen zu finden, sodass möglichst ähnliche einzelne Symbole oder ganze Abschnitte einander zugeordnet sind. Man versucht auf diese Weise funktionelle, strukturelle oder evolutionäre Ähnlichkeiten zu finden. Ein Beispiel für relevante strukturelle Ähnlichkeiten sind Proteinsequenzen. Wenn Proteine aus ähnlichen Aminosäuren in vergleichbaren Reihenfolgen aufgebaut sind, dann kann man davon ausgehen, dass diese auch eine ähnliche 3D-Struktur und ähnliche Funktionen haben, selbst wenn sie in unterschiedlichen Organismen vorkommen. Im Laufe der Evolution verändern sich aufgrund von Mutationen die DNA und die Proteine von Arten. Diese Vorgänge sind die Ursache für den beispiellosen Reichtum an Lebewesen auf der Welt und mit Hilfe von \emph{Sequenzalignments} kann man nachvollziehen wie diese Entwicklung vonstatten gegangen ist. Zu den häufigsten Mutationsarten bei Genmutationen (im Gegensatz zu Genommutationen und Chromosomenmuationen) gehören Punktmutationen, bei denen eine einzelne DNA-Base durch eine andere ersetzt wird, sowie Deletionen und Insertionen, bei denen ganze Abschnitte einer Sequenz gelöscht oder eingefügt wurden. Glücklicherweise haben unsere \emph{Alignments} Möglichkeiten genau diese Situationen abzubilden. Informell könnte man sagen, dass man bei einer Zuweisung die Elemente der Eingabestrings einander so zuordnet, dass jedem Symbol genau ein Symbol jeder anderen Sequenz oder eine neu eingefügte Lücke, Gap genannt, zugeordnet ist. Dabei darf die Reihenfolge der Elemente nicht verändert werden.

Betrachten wir dazu zwei kleine Beispiele von Wörtern, die häufig falsch geschrieben werden:

\ttfamily
\begin{center}
\begin{tabular}{ccc}
		OR-GINAL & \hspace{2cm} & SYLVESTER \\
		ORIGINAL & \hspace{2cm} & SILVESTER
\end{tabular}
\end{center}
\normalfont

Im ersten Fall wurde bei der falschen Schreibweise ein benötigter Buchstabe weggelassen. Damit es trotzdem zu einer passenden Zuordnung der anderen Buchstaben kommt, wurde in die erste Sequenz eine Lücke (-) eingefügt. Im evolutionären Kontext wäre dies ein Beispiel für eine Deletion. Im zweiten Beispiel wurde ein Buchstabe durch einen anderen, fehlerhaften ersetzt. Das ist ein klassisches Beispiel für eine Punktmutation oder einen Einzelnukleotid-Polymorphismus.

Die Berechnung eines \emph{Sequenzalignments} ist in vielen Fällen der erste Schritt einer \emph{Sequenzanalyse} in der Molekularbiologie \cite{cpm10}. Diese Analysen dienen unter anderem dazu zu bestimmen, ob Sequenzen miteinander verwandt sind (\emph{Homologie}), zum Bestimmen von Markergenen oder um direkt von der Sequenz auf die molekulare Struktur zu schließen. 

Ein zweites großes Einsatzgebiet des \emph{multiple sequence Alignment}-Problems ist das \emph{Natural Language Processing}, also der maschinellen Verarbeitung menschlicher Sprache \citep{s10}. Sätze, Wörter oder Ausdrücke können aligniert werden, um mechanisch Sätze zu übersetzen oder Texte zusammenzufassen. Noch einen Schritt weiter geht das alignieren von \emph{Phonemen}, wo es beispielsweise darum geht von der textuellen Darstellung auf die Aussprache zu schließen oder andersrum Sprache textuell darzustellen. Obwohl viele dieser Anwendungen nur auf zwei Sequenzen arbeiten, wie beispielsweise einen Text und seiner Übersetzung, gibt es auch Fälle bei denen multiple \emph{Alignments} nötig sind. Dazu gehören unter anderem Vergleiche von Texten, die in anderen Worten den selben Inhalt wiedergeben, oder von gleichbedeutenden Worten aus unterschiedlichen Sprachen der selben Sprachfamilie. Bei der Sprachentwicklung gibt es interessante Parallelen zu den evolutionären Vorgängen in Genomen. Im Kontext dieser Bachelorarbeit werden wir uns jedoch im Folgenden auf Anwendungen in der molekularen Bioinformatik beschränken.  

\section{Der Algorithmus von Needleman-Wunsch}

Einer der ersten Algorithmen zur Berechnung von paarweisen \emph{Alignments} war der von \cite{nw70}. Dieser weist Zuweisungen auf Symbolebene zwischen zwei Sequenzen Werte zu und das Ziel ist es mit Hilfe von dynamischer Programmierung die Summe dieser Werte zu maximieren. Wir werden in den nächsten Abschnitten zunächst das Paradigma der dynamischen Programmierung kennenlernen, bevor wir es beim Algorithmus von Needleman-Wunsch benutzen. Unter Verwendung des Algorithmus lässt sich danach die Komplexität des \emph{multiple sequence Alignment}-Problems verdeutlichen, die uns den Anlass gibt in den nächsten Kapiteln zwei leistungsstarke Heuristiken zu betrachten.

\subsection{Dynamische Programmierung}

Die dynamische Programmierung ist ein Prinzip zum algorithmischen Lösen eines Optimierungsproblems. Dazu wird ein größeres Problem unter Zuhilfenahme der Lösungen von sich überschneidenden Teilproblemen gelöst, wobei die zuvor berechneten Lösungen in einer Tabelle gespeichert werden. Durch die Wiederverwendung der bereits gelösten Teilprobleme lässt sich auf diese Weise die oft exponentielle Laufzeit eines naiven Algorithmus auf polynomielle Laufzeit verringern. Dabei muss man beachten, ob es für den jeweiligen Kontext angemessen ist den höheren Speicherbedarf aufgrund der zu speichernden Werte für die verbesserte Laufzeit in Kauf zu nehmen. Oft wird die dynamische Programmierung mit dem \emph{divide and conquer}-Prinzip verwechselt. Beide Paradigmen darauf basieren ein Problem in kleinere Teilprobleme zu zerlegen. Der Hauptunterschied zwischen ihnen ist, dass das Problem bei \emph{divide and conquer} in disjunkte Teilprobleme zerlegt wird, statt in sich überlappende \citep{clrs09}. Denken wir beispielsweise an Mergesort zurück: hier sortiere ich ein Feld, indem ich immer größer werdende disjunkte Teilfelder miteinander verschmelze.

In der Regel stellt man bei der dynamischen Programmierung zunächst eine Rekursionsgleichung auf mit deren Hilfe sich das Problem beschreiben lässt. Im Gegensatz zur \emph{Memoisation} wird diese Rekursionsgleichung jedoch nicht direkt und \emph{top-down} umgesetzt. Stattdessen werden \emph{bottom-up} zunächst die Basisfälle berechnet und darauf aufbauend immer größere Teilprobleme gelöst. Die Berechnung erfolgt dabei aber nichtsdestoweniger der Rekursionsformel entsprechend. Aufgrund der direkten Umsetzung der Rekursionsgleichung sind die Korrektheitsbeweise der Algorithmen der dynamischen Programmierung oft sehr einfach und aus der Korrektheit der Formel folgt meistens automatisch die des ganzen Algorithmus.

Dynamische Programmierung wurde in den 1940er-Jahren von Richard Bellman in Stanford und später bei der Denkfabrik RAND Corporation entwickelt. Nach ihm ist auch das \emph{Bellmannsche Optimalitätsprinzip} benannt, das besagt, dass man bei vielen Optimierungsproblemen die optimalen Lösungen von Teilproblemen benutzen kann, um die optimale Lösung des eigentlichen Problems zu berechnen. Warum Bellman den Begriff \enquote{dynamische Programmierung} gewählt hat, ist unklar. In seiner Autobiographie \enquote{Eye of the Hurricane} erklärt er \citep{b84}:

\begin{quotation}
	An interesting question is, Where did the name, dynamic programming, come from? The 1950s were not good years for mathematical research. We had a very interesting gentleman in Washington named Wilson. He was Secretary of Defense, and he actually had a pathological fear and hatred of the word research. I’m not using the term lightly; I’m using it precisely. His face would suffuse, he would turn red, and he would get violent if people used the term research in his presence. You can imagine how he felt, then, about the term mathematical. The RAND Corporation was employed by the Air Force, and the Air Force had Wilson as its boss, essentially. Hence, I felt I had to do something to shield Wilson and the Air Force from the fact that I was really doing mathematics inside the RAND Corporation. What title, what name, could I choose? In the first place I was interested in planning, in decision making, in thinking. But planning, is not a good word for various reasons. I decided therefore to use the word “programming”. I wanted to get across the idea that this was dynamic, this was multistage, this was time-varying. I thought, let's kill two birds with one stone. Let's take a word that has an absolutely precise meaning, namely dynamic, in the classical physical sense. It also has a very interesting property as an adjective, and that is it's impossible to use the word dynamic in a pejorative sense. Try thinking of some combination that will possibly give it a pejorative meaning. It's impossible. Thus, I thought dynamic programming was a good name. It was something not even a Congressman could object to. So I used it as an umbrella for my activities.
\end{quotation}

Ob diese Geschichte wirklich war ist, ist fraglich, denn die erste Arbeit Bellmans, die den Begriff benutzt, wurde bereits 1952 veröffentlicht, obwohl der oben genannte Charles Wilson erst ein Jahr später Verteidigungsminister wurde. Anderen Aussagen zufolge wurde der Begriff analog zur linearen Programmierung von George Dantzig gewählt, dem Erfinder des Simplex-Verfahrens \citep{rn09}. Dieser war ungefähr zur selben Zeit bei RAND beschäftigt. Das Adjektiv \enquote{dynamisch} bezieht sich dabei auf die künstliche Einfügung von Zeit in ein statisches Problem \citep{b57}. Genauer gesagt ist damit gemeint, dass die Reihenfolge der Durchführung von Berechnungen von Bedeutung ist: zunächst müssen die optimalen Lösungen der sich überlappenden Teilprobleme berechnet werden, bevor ich diese zur optimalen Lösung des eigentlichen Problems zusammensetzen kann.

\subsection{Der Algorithmus}

Der Algorithmus von Needleman-Wunsch ist ein \emph{Alignierer} auf der Basis von einzelnen Symbolen \citep{nw70}. Da dieses Kapitel einen eher motivierenden Charakter hat, werden wir den Algorithmus auf einer etwas informaleren Ebene betrachten. Zunächst lernen wir dafür eine etwas einfachere Definition eines \emph{Alignments} kennenlernen, die wir im nächsten Kapitel durch eine komplexere, aber äquivalente ergänzen, die sich insbesondere auch für multiple eignet, die aus mehr als zwei Sequenzen bestehen.

\begin{definition}[Alignment (NW)]
	Seien $S_1[1,\dots, n]$ und $S_2[1,\dots,m]$ zwei Zeichenketten über einem endlichen Alphabet. Eine Relation $\mathcal{R}$ zwischen Symbolen dieser beiden Sequenzen ist genau dann ein \emph{Alignment}, wenn für alle Paare von Zweiertupeln $S_1[i],S_2[j]$ und $S_1[i'],S_2[j']$ aus $\mathcal{R}$ gilt, dass aus $i < i'$ auch $j < j'$ folgt. Mit anderen Worten: es gibt keine überkreuzten Zuweisungen.
\end{definition}

Unser Ziel ist es jetzt aus allen möglichen \emph{Alignments} von $S_1$ und $S_2$ das beste zu finden. Dazu brauchen wir eine Möglichkeit die Güte von \emph{Alignments} zu vergleichen. Wenn zwei Symbole $s = S_1[i]$ und $s' = S_2[j]$ einander zugewiesen werden, dann weisen wir diesen ein Gewicht $\alpha_{s,t}$ zu, abhängig von der Ähnlichkeit der beiden Symbole. Oft wird hier nur zwischen Übereinstimmungen und Abweichungen unterschieden, aber in manchen Kontexten wie beispielsweise Proteinsequenzen mag es auch andere Möglichkeiten geben. Genauer widmen wir uns diesem Thema in Abschnitt \ref{subsec:subs_matr}. Wird hingegen eine Lücke in eine der beiden Sequenzen eingefügt, dann ziehen wir dann Kosten in der Höhe $\delta$ ab (sog. \emph{Gap Penalty}).

\begin{definition}{Score (NW)}
	Gegeben seien zwei Sequenzen $S_1$ und $S_2$ und ein \emph{Alignment} $\mathcal{R}$, sowie die \emph{Ähnlichkeitswerte} $\alpha$ und  der \emph{Gap Penalty} $\delta$. Dann definieren wir:
	
	\begin{equation}
		score(\mathcal{R},S_1,S_2) \coloneqq \sum_{(s,t)\in \mathcal{R}}{\alpha_{s,t}} - \sum_{s : \nexists t : (s,t)\in \mathcal{R}}{\delta} - \sum_{t : \nexists s : (s,t)\in \mathcal{R}}{\delta}
	\end{equation}
\end{definition}

Der \emph{Score} ist für ein \emph{Alignment} beim Algorithmus von Needleman-Wunsch also als die Summe aller \emph{Ähnlichkeitswerte} von einander zugewiesenen Symbolen der beiden Sequenzen definiert, von denen die \emph{Gap Penalties} aller eingefügten Lücken abgezogen werden. Das Ziel wird es jetzt sein eine Rekursionsgleichung aufzustellen mit deren Hilfe man diesen \emph{Score} maximieren kann. 

\section{Komplexität}

\chapter{Methodik}
\label{ch:methodik}

Im Methodik-Kapitel werden die mathematischen Ausführungen des Verfahrens bzw. des Algorithmus vorgestellt, jedoch technische Details zur Umsetzung und Implementierung (falls nötig) auf ein darauffolgendes Kapitel verlagert.


\section{Mathematische Notation}
\label{s:notation}

Mathematische Formeln können mittels der \verb+\begin{align}...\end{align}+ Umgebung gesetzt werden:

\begin{align}
f(n) & =
	\begin{cases}
		n/2, & \text{wenn }n\text{ gerade,}\\
		3n+1, & \text{wenn }n\text{ ungerade.}
	\end{cases}
\label{eq:f} \\
%
g(n) & = \frac{n}{2} \label{eq:g}
\end{align}

\section{Algorithmus}
\label{s:algorithmus}

Eigene Algorithmen beschreibt man am Besten mit Hilfe von Pseudo-Code und dem Paket \verb+algorithm+.

\begin{algorithm}
\caption{Algorithmus}
\label{alg:alg}
\begin{algorithmic}
\algsetup{indent=2em}

\REQUIRE Argument $n\in\mathbb{N}$
\STATE $a = 0$
\FOR{ $i=0,\dots,n$}
	\STATE $a = a + 1$
\ENDFOR
\RETURN $a$
\end{algorithmic}
\end{algorithm}

\chapter{Ergebnisse}
\label{ch:ergebnisse}

Dieses Kapitel sollte die Ergebnisse beinhalten, die mit den Methoden aus \autoref{ch:methodik} erstellt wurden.

\begin{table}[h]
\caption{Beispieltabelle}
\begin{center}
	\begin{tabular}{|c||c|c|}
		\hline
		Spalte1 & Spalte2 & Spalte3 \\ 
		\hline\hline
		   1    &    2    &    3    \\ 
		\hline
	\end{tabular}
\end{center}
\label{tbl:table}
\end{table}

\chapter{Diskussion}
\label{ch:diskussion}

In diesem Kapitel werden die zuvor vorgestellten Ergebnisse der Arbeit diskutiert. Häufig wird es mit dem Fazit zusammengelegt.

\chapter{Fazit}
\label{ch:fazit}

\section{Zusammenfassung}

Nachdem wir mit dem Algorithmus von Needleman-Wunsch einen ersten Einblick in Sequenzalignments bekommen haben, wurden im Laufe dieser Arbeit zwei algorithmische Ansätze für das Multiple-Sequence-Alignment-Problem vorgestellt: DIALIGN 2.2 von Morgenstern et al. und der Min-Cut-Ansatz von Corel et al.

DIALIGN 2.2 mit seinen Erweiterungen berechnet solide globale Alignments und ist bei denen von lokalen Sequenzfamilien überragend. Mit Hilfe von dynamischer Programmierung berechnet das speicherplatzeffiziente Verfahren aus DIALIGN 2.2 die quadratisch vielen paarweisen Alignments. Dabei wird jedem möglichen Fragment mit Hilfe der Gewichtsfunktion $w^{*}$ aus DIALIGN 2.0 ein Gewicht zugewiesen und das Gesamtgewicht aller miteinander konsistenten Fragmente maximiert. Durch eine Einschränkung der zu speichernden Fragmente, konnte der benötigte Speicherplatz weiter gesenkt werden. Um für unser multiples Alignment Fragmente zu bevorzugen, die Übereinstimmungen mit ähnlichen Abschnitten in weiteren Sequenzen haben, berechnen wir im nächsten Schritt Überlappgewichte. Anstelle der naiven Berechnung, die alle Fragmente miteinander vergleicht, habe ich einen effizienteren Algorithmus entwickelt, der auf dem parallelen Traversieren der Listen von Fragmenten aller paarweisen Alignments basiert.

Nachdem die Überlappgewichte berechnet wurden, nutzen wir diese für die Sortierung aller Fragmente. Eine gierige Heuristik wählt jetzt solange das Fragment mit dem höchsten Gewicht, das zu allen zuvor gewählten konsistent ist, bis keine mehr übrig sind. Für die rechenintensive Berechnung der Konsistenzgrenzen haben wir den mit DIALIGN 2.1 eingeführten Ansatz von Abdedda\"im betrachtet. Dieser berechnet mit Hilfe eines Spanning Set of disjoint Paths (SSDP) die transitive Hülle des sogenannten Alignmentgraphen, über den sich die Konsistenzgrenzen berechnen lassen.
Solange sich das Alignment durch Fragmente mit positiven Gewichten erweitern lässt, führen wir auf den Teilsequenzen zwischen den Konsistenzgrenzen weitere Durchläufe von DIALIGN durch. Schlussendlich fügen wir für unsere Ausgabe Lücken in die Menge von alignierten Sequenzen ein, sodass alle Zuweisungsspalten genau untereinander stehen.
	
Der im Kern auf DIALIGN basierende Ansatz von Corel et al. erweitert das DIALIGN-Verfahren um graphentheoretische Ansätze \cite{cpm10}. Der Hauptunterschied zum ursprünglichen Verfahren ist, dass eine Menge von multiplen Zuweisungen auf eine konsistente Menge verringert wird, statt nach und nach mit paarweisen Zuweisungen ein Alignment aufzubauen. Zunächst werden dafür die Verbindungen aller paarweisen Alignments in einem Inzidenzgraphen gesammelt. In diesem Graphen fassen wir Zusammenhangskomponenten als Flussnetzwerke auf und trennen mit Hilfe eines Algorithmus zur Berechnung von minimalen Schnitten Knoten, die direkt oder indirekt miteinander verbunden sind und gleichzeitig aus der selben Sequenz kommen. Wurden alle dieser Mehrdeutigkeiten aufgelöst, ergibt sich als Zwischenstand eine Menge von partiellen Zuweisungsspalten. In diesen gibt es keine transitiven Mehrfachzuweisungen mehr, aber Überkreuzungen können weiterhin vorkommen. 

Zur Entfernung von Stellen aus Zuweisungen, die für Inkonsistenzen sorgen, haben wir den Algorithmus von Pitschi kennengelernt. In einem Sukzessionsgraphen fügen wir die partiellen Zuweisungsspalten als Knoten ein, die miteinander verbunden sind, wenn es Sequenzen gibt, in denen Stellen der Knoten Nachfolger voneinander sind. Für jede Sequenz $S_i$ konstruieren wir basierend darauf einen neuen Graphen in dem Ketten von Knoten auf einem Pfad vom Start- zum Endknoten einer Menge von konsistenten Zuweisungen bezüglich $S_i$ entsprechen. Weil wir diese Menge maximieren möchten, wählen wir für jede Sequenz den Pfad maximaler Länge durch den entsprechenden Graphen.

Die resultierende Menge von konsistenten Zuweisungsspalten können wir wie zuvor bei DIALIGN in unseren Alignmentgraphen einfügen und danach auf den Teilsequenzen weitere Fragmente berechnen. Auch die Vorbereitung der Ausgabe erfolgt analog, nachdem sich das Alignment nicht mehr vergrößern lässt.

Aufgrund der begrenzten Zeit der Arbeit und dem großen Umfang konnte nicht das ganze Verfahren implementiert werden. Stattdessen wurde exemplarisch ein Algorithmus zur Bestimmung des längsten Pfades in der Programmiersprache C++ und mit Hilfe der Bibliothek \emph{Boost Graph Library} umgesetzt. Dieser Algorithmus stellt einen wichtigen Schritt im Algorithmus von Pitschi dar, den wir zuvor auf einer höheren Abstraktionsebene betrachtet haben. 

\section{Weiterführen Arbeiten} \label{sec:fut_work}

Im Folgenden gehen wir auf mögliche Weiterentwicklungen des vorgestellten Verfahrens ein. Teilweise wurden diese bereits im Lauf der Bachelorarbeit vorgestellt, es fehlt aber noch eine Evaluierung der Ergebnisse mit Hilfe einer vollständigen Implementierung des Algorithmus. Andere potentielle Verbesserungen wurden hingegen noch nicht erwähnt.

\cite{cpm10} haben festgestellt, dass die numerischen Scores ihres Verfahrens in vielen Fällen schlechter waren als die von DIALIGN 2.2, obwohl die Ergebnisse biologisch relevanter waren. Sie schlussfolgern daher, dass es auf Basis der aktuellen Gütefunktion keine substantielle Verbesserung des DIALIGN-Ansatzes mit Hilfe neuartiger Heuristiken mehr geben wird. Eine erste mögliche Verbesserung des Verfahrens könnten stochastische Ansätze für die Gütefunktion sein, wie beispielsweise \emph{Conditional Random Fields} oder \emph{Hidden Markov Models}\cite{sm10,e95}. Außerdem könnte man statt paarweisen Alignments auch kleinere multiple Zuweisungen als Grundbausteine für die Heuristiken verwenden. Ein Ansatz sind zum Beispiel ternäre Alignments, bei denen sich Fragmente gleich über alle drei der betrachteten Sequenzen ziehen können. Leider sorgten sowohl die größere Anzahl dieser Alignments ($\oh(n^3)$ statt $\oh(n^2)$), als auch die höhere Laufzeit pro Tripel von Sequenzen (vermutlich etwa $\oh(L^3)$ statt $\oh(L^2)$) für eine insgesamt erheblich höhere Komplexität. Erst aufwändige Implementierungen eines darauf basierenden Verfahrens dürften zeigen, ob die zu erwartenden besseren Ergebnisse den Aufwand rechtfertigen.

\cite{snkm04} haben mit DIALIGN P eine parallelisierte Implementierung von DIALIGN entwickelt. Aber auch beim Min-Cut-Ansatz lassen sich viele Schritte des Algorithmus leicht parallelisieren:

\begin{enumerate}[topsep=0pt,itemsep=-1ex,partopsep=1ex,parsep=1ex]
	\item Wie bei DIALIGN P kann man die die voneinander unabhängigen paarweisen Alignments gleichzeitig berechnen.
	\item Die Überlappgewichte lassen sich parallel bestimmen, wenn man für die Fragmente jedes paarweisen Alignments nur lesend auf die der anderen Alignments zugreift, bei denen eine der zwei selben Sequenzen beteiligt war.
	\item Alle chnitte sind voneinander unabhängig. Sie können also in jedem Durchlauf der while-Schleife für jede Zusammenhangskomponente im Inzidenzgraphen gleichzeitig berechnet werden.
	\item Für alle Sequenzen $S_i \in S$ können beim Algorithmus von Pitschi simultan die Graphen $G_{S_i}$ konstruiert und die längsten Pfade bestimmt werden.
\end{enumerate}

\noindent Auf diese Art und Weise ließe sich die mitunter sehr lange Laufzeit des graphtheoretischen Ansatzes vermutlich beträchtlich verringern.

Für den Algorithmus \ref{alg:speichereffizient} habe ich einige Ansätze dafür gefunden, den benötigten Speicherplatz noch weiter zu verringern. Ein Beispiel ist die Verbesserung, dass nicht für jeden Endpunkt $(i,j)$ alle dort endenden Fragmente gespeichert werden, sondern nur das, welches zum aktuellen Zeitpunkt das maximale Präfixgewicht hat. Die Speicherung aller Fragmente ist unnötig, weil später ohnehin nur das Maximum bestimmt wird.

Beim Konstruieren des Alignmentgraphen könnte man, anstatt zwei antiparallele Kanten zwischen den neu alignierten Stellen einzufügen, auch eine Verschmelzung von ihnen in einem Knoten durchführen. Das entspricht dann nicht genau dem Erhalt der transitiven Hülle auf einem Graph mit SSDP, weil die Pfade entlang jeder Sequenz durch den Graphen nicht mehr disjunkt sind. In der Praxis wäre das aber kein Problem, weil die Doppelkanten ohnehin für Verbindungen in beide Richtungen sorgen und die Knoten insofern äquivalent im Sinne der Funktion \textrm{EdgeAddition} sind. Das ist auch wenig überraschend, weil ein Alignment eine Äquivalenzrelation ist, in der alignierte Stellen Teil der selben Äquivalenzklasse sind. Man müsste in der Praxis testen wie sich der verringerte Speicherverbrauch durch die kleinere Anzahl an Kanten im Vergleich zu den zusätzlichen Kosten für die Vereinigung der ausgehenden Kanten auswirkt.

Wie Abschnitt \ref{subsec:einsparungen} schon angedeutet könnte man die Felder nextClass[] und prevClass[] auch durch effizientere Datenstrukturen ersetzen. Es ließe sich viel Speicherplatz sparen, wenn man nicht für alle Waisen Vorgänger und Nachfolger speichert, sondern stattdessen für die Äquivalenzklassen an Waisen zwischen zwei alignierten Stellen. Denkbar wäre zum Beispiel eine Waldstruktur mit einem Suchbaum für jede Sequenz, der als Knoten Angaben über Start- und Endpunkte solcher Klassen enthält.

Im Abschnitt über den Algorithmus von Pitschi haben wir gezeigt, dass es Situationen geben kann, in denen die einfache Heuristik zum Löschen von Kanten im Graphen nicht das erwünschte Ergebnis liefert. Beim Versuch, die Zyklen im Graphen aufzulösen, kann es dazu kommen, dass auch eine Vielzahl an unbeteiligten Kanten gelöscht werden, im schlimmsten Fall sogar alle. Ein Beispiel für eine solche Situation sind Sequenzfamilien, bei denen es im Lauf der Evolution zu einer Permutation eines größeren Abschnitts gekommen ist. Um solchen Fällen vorzubeugen kann man beispielsweise überprüfen, ob die ursprüngliche Heuristik mehr als beispielsweise 25\% aller Kanten löscht, und falls das der Fall ist, auf einen anderen Algorithmus wie den von \cite{els93} zurückgreifen. Das wäre zwar zusätzlicher Aufwand, aber besser als die teuer berechneten partiellen Zuweisungsspalten nicht zu benutzen.

In Abschnitt \ref{sec:ueberlapp} habe ich einen effizienten Algorithmus zum Berechnen der Überlappgewichte vorgestellt. Statt alle $\oh(n^2\cdot L)$ Fragmente miteinander auf Überschneidungen zu überprüfen, werden nur Fragmente verglichen, an denen genau eine gemeinsame Sequenz beteiligt ist. Weil die Listen von Fragmenten nach dem Berechnen der paarweisen Alignments sortiert sind, lässt sich der Aufwand dafür weiter verringern, indem wir parallel über sortierte Listen traversieren.

Anstatt einheitliche Kapazitäten in den Flussnetzwerken über den Inzidenzgraphen zu benutzen, verwende ich Kapazitäten, die von der Ähnlichkeit der verbundenen Stellen abhängen. Das Kalkül dahinter ist, dass im Zweifelsfall eher unähnliche Symbole voneinander getrennt werden, anstelle von ähnlichen. Ob dieses Verfahren eine merkliche Verbesserung der Ergebnisse bewirkt muss man in der Praxis testen.

Diese potentiellen Verbesserungen werden in Zukunft genauer analysiert und implementiert. Es erscheint vielversprechend, dass sie eine Weiterentwicklung des DIALIGN-Verfahrens für das Multiple-Sequence-Alignment darstellen.















% Literaturverzeichnis
\bibliographystyle{unsrtdin}
\bibliography{Quellen}

% Eidesstattliche Erklärung
\chapter*{Eidesstattliche Erklärung}

% Die Aktuelle Version der Eidesstättlichen Erklärung kann beim zuständigen Prüfungsamts gefunden werden.
% Nachfolgend ist die Version für Bachelor und Master des Prüfungsamts Mathe/Informatik vom 05.10.2016

Hiermit versichere ich, dass die vorliegende Arbeit über \textit{\glqq Titel\grqq} selbstständig verfasst worden ist, dass keine anderen Quellen und Hilfsmittel als die angegebenen benutzt worden sind und dass die Stellen der Arbeit, die anderen Werken – auch elektronischen Medien – dem Wortlaut oder Sinn nach entnommen wurden, auf jeden Fall unter Angabe der Quelle als Entlehnung kenntlich gemacht worden sind.

\vspace{1cm}

\parbox{20em}{\hrulefill}

Vorname Nachname, Münster, \today

\vspace{1cm}

Ich erkläre mich mit einem Abgleich der Arbeit mit anderen Texten zwecks Auffindung von Übereinstimmungen sowie mit einer zu diesem Zweck vorzunehmenden Speicherung der Arbeit in eine Datenbank einverstanden.

\vspace{1cm}

\parbox{20em}{\hrulefill}

Vorname Nachname, Münster, \today

\end{document}