%!TEX TS-program = pdflatex
\documentclass[wide,xcolor={x11names},hyperref={colorlinks=false},pantone312]{beamer}

% Basics und Codierung
% ===========================================================
\usepackage{wwustyle2}
\usepackage[ngerman]{babel}
\usepackage[utf8]{inputenc}
\usepackage[T1]{fontenc}
\usepackage[german=quotes]{csquotes}
\usepackage{scrtime}
\usepackage{etex}
\usepackage{shellesc}
% ===========================================================


% Fonts und Typographie
% ===========================================================
\usepackage{sourcecodepro}
\usepackage[default]{sourcesanspro}
\usepackage{nimbusmononarrow}
\usepackage{ellipsis}
\newcommand{\bet}[1]{\textbf{\color{maincolor}#1}}
\newcommand{\minor}[1]{\textcolor{black!50}{#1}}
\newcommand{\minoritem}{\item[{\footnotesize \textcolor{black!50}{$\blacktriangleright$}}]}
\newcommand{\code}[1]{\texttt{#1}}
\usepackage{xspace}
\makeatletter 
\xspaceaddexceptions{\grqq \grq \csq@qclose@i \} } 
\makeatother
\usefonttheme[onlymath]{serif}
\usepackage{multicol}
% ===========================================================

% Farben	
% ===========================================================
	\usepackage{xcolor}
	\definecolor{fbblau}{HTML}{3078AB}
	\definecolor{mediumgray}{gray}{.65}
	\definecolor{blackberry}{rgb}{0.53, 0.0, 0.25}
% ===========================================================

	
% Mathe-Pakete
% ===========================================================
	\usepackage{mathtools}
	\usepackage{amssymb}
	\usepackage[bigdelims]{newtxmath}

	\input{MathCmds.tex}
	\usepackage{wasysym}
	
\newcommand{\zerodisplayskips}{%
%\setlength{\abovedisplayskip}{0pt}%
%\setlength{\belowdisplayskip}{0pt}%
%\setlength{\abovedisplayshortskip}{0pt}%
%\setlength{\belowdisplayshortskip}{0pt}
%\setlength{\multicolsep}{0pt}
}
\appto{\normalsize}{\zerodisplayskips}
\appto{\small}{\zerodisplayskips}
\appto{\footnotesize}{\zerodisplayskips}
% ===========================================================

% TikZ
% ===========================================================
	\usepackage{tikz}
	\usepackage{tikz-cd}					% kommutative Diagramme
	\usetikzlibrary{arrows.meta}			% mehr Pfeile!
	\usetikzlibrary{shadows}
	\usetikzlibrary{calc}
	\tikzset{>=Latex}						% Standard-Pfeilspitze
	\usetikzlibrary{automata,positioning}
	\usetikzlibrary{matrix}
% ===========================================================

% minted
% ===========================================================
\usepackage{minted}
\setminted{%
	style=default,
	fontsize=\footnotesize,
	breaklines,
	breakanywhere=false,
	breakbytoken=false,
	breakbytokenanywhere=false,
	breakafter={.,},
	autogobble,
	numbers=left,
	numbersep=3mm,
	tabsize=4,
	frame=lines
}
\setmintedinline{%
	fontsize=\normalsize,
	numbers=none,
	numbersep=12pt,
	tabsize=4,
	%bgcolor=gray!15,
}
% ===========================================================
\author{Joschka Strüber (\texttt{j.st@wwu.de})}
\title{Ein graphtheoretischer Ansatz für das \\ \textit{multiple sequence Alignment}-Problem}
\subtitle{\glqq Effiziente Algorithmen\grqq{} in einer Bachelorarbeit}
\date{\today}
\begin{document}
\setbeamertemplate{section in toc}[sections numbered]

\begin{frame}[plain]
  \maketitle
\end{frame}

\begin{frame}[t]{}
\tableofcontents[hidesubsections, hideothersubsections]
\end{frame}

\AtBeginSection[]
{
	\begin{frame}[t]
		\tableofcontents[currentsection, hidesubsections, hideothersubsections,sectionstyle=show/shaded]
	\end{frame}
}

\section{Motivation}

\begin{frame}[t]{Das \textit{multiple sequence Alignment}-Problem}
	\begin{itemize}
		\item Gegeben ist eine Menge von Zeichenketten.
		\item Beispiel:
		\begin{enumerate}
			\ttfamily
			\item TCGTCTGCACGCGCTCTGCGAT
			\item AGTCGTCTGCACGGGATCTGCGA
			\item AATAGTCATGGACGCGTGCTCTA
			\item ATAGTCATGGACGCGTGCGCGAT
			\normalfont
		\end{enumerate}
		\item Frage: Wie kann man einzelne Symbole oder ganze Abschnitte dieser Sequenzen einander zuordnen, sodass diejenigen Bereiche übereinander stehen, die sich möglichst ähnlich sind? \pause
		\item In unserem Beispiel könnte das beispielsweise so aussehen:
		\begin{enumerate}
			\ttfamily
			\item --TCGTC-TGCACGC--GCTCTGCGAT
			\item AGTCGTC-TGCACG-G-GATCTGCGA-
			\item AATAGTCATGGACGCGTGCTC---TA-
			\item -ATAGTCATGGACGCGTGCGC---GAT
			\normalfont
		\end{enumerate}		
	\end{itemize}
\end{frame}

\begin{frame}[t]{Wofür wird MSA benötigt?}
	\begin{figure}[htbp]
		\begin{minipage}[t]{5cm}
			\vspace{0pt}
			\centering
			\includegraphics[scale=0.35]{Spektrum}
		\end{minipage}
		\hfill
		\begin{minipage}[t]{7cm}
			\small
			\begin{itemize}
				\item Multiples Sequenzalignment wird hauptsächlich in der Bioinformatik benutzt.
				\item Verglichen werden üblicherweise DNA- oder Proteinsequenzen.
				\item Wird zum Beispiel benutzt, wenn man einen gemeinsamen evolutionären Ursprung zwischen mehreren Sequenzen vermutet, oder allgemein, um mögliche Resultate von Mutationen zu finden.
				\item Gefunden werden können beispielsweise Punktmutationen (Änderungen einzelner Basen oder Aminosäuren), sowie eingefügte oder gelöschte Abschnitte.
			\end{itemize}
			\normalsize
		\end{minipage}
	\end{figure}
\end{frame}

\begin{frame}[t]{Komplexität I}
	\begin{itemize}
		\item Problem: Unter allen auch nur halbwegs realistischen Annahmen ist das \textit{multiple sequence Alignment}-Problem NP-schwer mit Laufzeit $\oh(length^ n)$, wobei n die Anzahl der alignierten Sequenzen ist.
		\item Beweisskizze mit Hilfe des Algorithmus von Needleman-Wunsch:
		\item Kosten für alignieren von Zeichen $a$ mit Zeichen $b$ kostet $\alpha_{a,b}$, wobei typischerweise $\alpha_{a,a} = 0$ gilt.
		\item Gap Penalty $\delta$, wenn wir die beiden Zeichen nicht miteinander alignieren und stattdessen in eine der Sequenzen eine Lücke einfügen.
		\item Gesamtkosten lassen sich mit dynamischer Programmierung und dieser Rekursionsgleichung lösen: \\
			$O[i,j] \coloneqq min\{\alpha(X[i], Y[j]) + O[i-1, j-1], \delta + O[i-1, j], \delta + O[i, j-1]\}$
	\end{itemize}
\end{frame}

\begin{frame}[t, fragile]{Komplexität II}
	\footnotesize
	\begin{tikzcd}
		O[1,1] \arrow[r] \arrow[d] \arrow[rd] & ... \arrow[d] \arrow[rd] & ... \arrow[d] & O[1,m] \arrow[d] \\
		... \arrow[r] \arrow[rd] & O[i-1,j-1] \arrow{rd}[near start, sloped]{\alpha_{X[i],Y[j]}} \arrow{d}{\delta} \arrow{r}{\delta} & O[i-1,j] \arrow[r] \arrow[rd] \arrow{d}{\delta} & ... \\
		... \arrow[r] \arrow[rd] & O[i,j-1] \arrow[d] \arrow[rd] \arrow{r}{\delta} & O[i,j] \arrow[d] \arrow[rd] \arrow[r] & ... \arrow[d] \\
		O[n,1] \arrow[r] & ... & ... \arrow[r] & O[n,m]
	\end{tikzcd}
	\normalsize
	\begin{itemize}			
		\item Alignment mit minimalen Kosten entspricht Pfad mit geringsten Kosten durch diesen Pfad, der in $\oh(n*m)$ berechnbar ist.
		\item Vorgehen lässt sich auch mit n-vielen Sequenzen durchführen $\rightarrow$ Pfad durch n-dimensionale Matrix 
	\end{itemize}
\end{frame}

\section{Das DIALIGN-Verfahren mit einem Min-Cut-Ansatz für das Konsistenzproblem}

\begin{frame}[t]{Grundidee hinter DIALIGN}
\end{frame}

\begin{frame}[t]{Beispielsequenzen und Gewichtsfunktionen}
\end{frame}

\begin{frame}[t]{1. paarweises Alignment}
\end{frame}

\begin{frame}[t]{2. paarweises Alignment}
\end{frame}

\begin{frame}[t]{3. paarweises Alignment}
\end{frame}

\begin{frame}[t]{Konsistenz und Zwischenstand}
\end{frame}

\begin{frame}[t]{Gieriges multiples Alignment}
\end{frame}

\begin{frame}[t]{Flussnetzwerke}
\end{frame}

\begin{frame}[t]{Inzidenzgraph}
\end{frame}

\begin{frame}[t]{Minimaler Schnitt auf Zusammenhangskomponenten}
\end{frame}

\begin{frame}[t]{Beispiele nichtkonsistenter Zuweisungen}
\end{frame}

\begin{frame}[t]{Unser Graph mit gelöschten Kanten}
\end{frame}

\begin{frame}[t]{Sukzessionsgraph}
\end{frame}

\begin{frame}[t]{Zyklen im Sukzessionsgraph und wie wir diese entfernen}
\end{frame}

\begin{frame}[t]{Das Dilemma}
\end{frame}

\begin{frame}[t]{Erschummelter DAG}
\end{frame}

\begin{frame}[t]{Folge der längsten Kette}
\end{frame}

\begin{frame}[t]{Das resultierende multiple Alignment}
\end{frame}

\begin{frame}[t]{Vereinfachungen, die ich vorgenommen habe}
\end{frame}

\section{Umsetzung in der Bachelorarbeit}

\begin{frame}[t]{Gierige Algorithmen und Dynamische Programmierung}
	\begin{itemize}
		\item Gierige Algorithmen:
		\begin{itemize}
			\item Treffe lokal die bestmögliche Entscheidung und verwerfe diese danach nicht wieder
			\item Gute Laufzeit, aber liefern oft nicht das optimale Ergebnis ($\rightarrow$ gut geeignet für Heuristiken)
			\item Wird in DIALIGN beim Zusammensetzen des multiplen Alignments benutzt
			\item Beispiele: Algorithmus von Prim, Fractional Knapsack-Problem
		\end{itemize} \pause
		\item Dynamische Programmierung:
		\begin{itemize}
			\item Berechnung der optimalen Lösung durch Kombination von optimalen Lösungen sich überlappender Teilprobleme, wobei diese gespeichert werden und bei Bedarf abgerufen werden können
			\item Beispiele: Algorithmus von Needleman-Wunsch, Knapsack-Problem
		\end{itemize}
	\end{itemize}
\end{frame}

\begin{frame}[t]{Flussnetze und der Push-Relabel-Algorithmus}
	\begin{itemize}
		\item Flussnetze:
		\begin{itemize}
			\item gerichteter, gewichteter Graph mit ausgewiesener Quelle und Senke; Gewichte der Kante entsprechen maximaler Kapazität an Fluss, die über diese fließen können
			\item Ziel: Maximierung des Gesamtflusses eines Flussnetzwerkes von der Quelle zur Senke
			\item Min-Cut-Max-Flow-Satz: der Wert eines maximalen Flusses entspricht dem Wert des minimalen Schnitts $\rightarrow$ benötigen wir zum Auflösen von Inkosistenzen im Inzidenzgraphen 
		\end{itemize} \pause
		\item Push-Relabel-Algorithmus:
		\begin{itemize}
			\item Vorgehen:
			\item besitzt eine bessere Laufzeit, als der in der Veröffentlichung benutzte Edmonds-Karps-Algorithmus
		\end{itemize}
	\end{itemize}
\end{frame}

\begin{frame}[t]{Algorithmus im Detail}
	\begin{itemize}
		\item Gewichtsfunktionen für DNA- und Proteinsequenzen (BLOSUM62)
		\item Berechnung von paarweisen Alignments mit Hilfe von dynamischer Programmierung
		\item Aufbau des Inzidenzgraphen und Min-Cut auf den Zusammenhangskomponenten
		\item Aufbau des Sukzessionsgraphen und Löschen von Sites, um Konsistenz herzustellen
		\item Verwendung der so gefundenen alignierten Spalten als Ankerpunkte, um zwischen diesen das klassische DIALIGN laufen zu lassen
		\item Wenn mir am Ende noch langweilig ist: Blick auf DIALIGN TX, um dieses gegebenfalls zwischen den Ankerpunkten zu verwenden
	\end{itemize}
\end{frame}

\begin{frame}[t]{Programmierung}
	\begin{itemize}
		\item Programmiersprache C++
		\item Boost Graph Library, die bereits ausgeklügelte Graphen, Flussnetzwerke und Algorithmen zur Berechnung des maximalen Flusses bereitstellt
		\item variabelster Teil der Bachelorarbeit:
		\begin{itemize}
			\item bei wenig Zeit am Ende: Ein- und Ausgabe über simple Textdateien
			\item bei viel Zeit: Umsetzung mit Formaten wie Clustal und FASTA, sowie Visualisierung über externe Visualisierungssoftware
		\end{itemize}
	\end{itemize}
\end{frame}

\begin{frame}[t]{Beurteilung der Güte der Ergebnisse}
	\begin{itemize}
		\item Bewertung des Algorithmus ist schwierig, weil er definitiv nicht das perfekte Alignment berechnet, ich aber nicht das Expertenwissen besitze, um die Güte eines Alignments wirklich beurteilen zu können
		\item Test auf BAliBase für global und (D)IRMBASE für lokal verwandte Sequenzen $\rightarrow$ bereits ausgewertete Sequenzen für die ein richtiges Alignment bekannt ist
		\item Erfordert noch Einarbeitung
	\end{itemize}
\end{frame}

\end{document}
