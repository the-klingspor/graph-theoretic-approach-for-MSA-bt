\chapter{Ein Min-Cut-Ansatz für das Konsistenzproblem}
\label{ch:min-cut}
-skizzierter Ablauf des Algorithmus

\section{Flussnetzwerke}

\subsection{Einführung}

\subsection{Wichtige Algorithmen}

\subsection{Der \emph{Max-Flow-Min-Cut-Satz}}

\section{Inzidenzgraphen und das Auflösen von Inkosistenzen mit Hilfe von Flussnetzwerken}

\subsection{Komplexität}
$\oh(n^4*l^{7/2})$

\section{Sukzessorgraphen und der Algorithmus von Pitschi}
$\oh(n^3 * l)$ für n-faches Topological Sort auf einem Knoten mit $\oh(n^2 * l)$-vielen Kanten im schlimmsten Fall.

\section{Ankerpunkte}

\section{Gesamtkomplexität}
$\oh(n^4*l^{7/2})$