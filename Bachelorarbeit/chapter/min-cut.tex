\chapter{Ein Min-Cut-Ansatz für das Konsistenzproblem}
\label{ch:min-cut}
Da es, wie im letzten Kapitel gezeigt, auf manchen Sequenzfamilien durch die gierige Heuristik von DIALIGN zu suboptimalen \emph{Scores} und \emph{Alignments} kommt, werden wir jetzt einen verbesserten graphtheoretischen Ansatz von \cite{cpm10} betrachten.

Dazu benötigen wir zwei verschiedene Graphen: zum einen den \emph{Inzidenzgraphen}, bei dem alle \emph{Stellen} Knoten sind und ihre \emph{Anker} Zusammenhangskomponenten. Der zweite ist der \emph{Sukzessorgraph}, der die Zusammenhangskomponenten unseres \emph{Inzidenzgraphen} als Knoten und die natürliche Ordnung auf den Sequenzen als Kanten benutzt. Man kann sich vorstellen, dass genau dann eine Kante zwischen zwei  Diese beiden Datenstrukturen werden wir benutzen, um \emph{Inkonsistenzen} aufzulösen. Wenn wir uns an die Definition von \emph{Konsistenz} aus dem letzten Kapitel erinnern, dann stellen wir fest, dass es zwei Arten von ihr gibt: zum einen implizite, transitive Mehrfachzuweisungen bei denen einer \emph{Stelle} einer Sequenze mehrere \emph{Stellen} einer anderen Sequenz zugeordnet sind und zum anderen überkreuzte Zuweisungen.

Als Ausgangspunkt starten wir wieder mit unseren paarweisen \emph{Alignments} aus DIALIGN. \emph{Überlappgewichte} brauchen wir in unserem Fall nicht. Dann konstruieren wir mit Hilfe dieser Zuweisungen unseren \emph{Inzidenzgraphen} und benutzen einen Algorithmus zur Berechnung des minimalen Schnitts (\enquote{min-cut}) auf den Zusammenhangskomponenten, um alle \emph{Inkonsistenzen} aufgrund von transitiven Mehrfachzuweisungen aufzulösen. Die so entstehenden Zusammenhangskomponenten benutzen wir, um einen \emph{Sukzessorgraphen} aufzubauen. Dank eines Algorithmus von \cite{pdc10} können wir mit diesem Überkreuzungen aus unserer Relation löschen. Alle dieser Konzepte werden wir im Laufe dieses Kapitel formal definieren, genauer analysieren und die Korrektheit der Aussagen beweisen.
 
\section{Flussnetzwerke}

\subsection{Einführung}

Stell dir eine Produktionsstätte vor, wo ein bestimmtes Produkt hergestellt wird und eine Verwendungsstätte bei der das Produkt benötigt wird. Zwischen dem Ort der Produktion und dem Ort der Verwendung gibt es ein Schienennetz über das die Produkte geliefert werden können. Über jeden Abschnitt der Schienen kann aber nur eine bestimmte Anzahl an Waggons geleitet werden. So mag es weniger stark befestigte Strecken geben, wo nur kleinere oder leichtere Züge fahren können und andere gut ausgebaute mit mehreren Gleisen nebeneinander. Die Anzahl an Einheiten, die über einen Abschnitt geleitet werden können, nennt man Kapazität, während Produktionsstätte und Zielort Quelle und Senke heißen. Ziel ist es zu bestimmen wie viele Produktionseinheiten über dieses Netz von der Fabrik zum Verbraucher geliefert werden können.

Neben diesem Problem kann man mit Flussnetzwerken noch viele andere Anwendungen modellieren, zum Beispiel eine chemische Produktionsstraße mit vielen Rohren, die jeweils nur den Durchfluss einer bestimmten Menge erlauben. Oder Stromnetze mit Kraftwerken und Endverbrauchern zwischen denen es Stromleitungen gibt, die jeweils höchstens einen bestimmten Stromfluss erlauben. Für eine Hochspannungsleitung mag dieser sehr hoch und für die Leitungen, die direkt zu den Häusern gehen, sehr klein sein. Stromnetze bieten uns auch eine direkte Analogie, wie mit den Knotenpunkten zwischen den Verbindungen umzugehen ist. An jedem Knoten, außer der Quelle und Senke, wird genauso viel Fluss hinein- wie wieder hinausgeleitet. Es wird nichts gespeichert oder geht verloren. Dieses Konzept, das man \emph{Flusserhaltung} nennt, funktioniert genau wie das erste \emph{Kirchhoffsche Gesetz} bei Strömen \citep{clrs09}. 

Ursprünglich stammt das Konzept des Flussnetzwerkes aus dem Kalten Krieg und der militärischen Forschung. Es wurde das Schienennetzwerk des Sowjetunion in Osteuropa untersucht, um herauszufinden wie viel Material die UdSSR im Kriegsfall aus dem russischen Kerngebiet nach Mitteleuropa transportieren könnte und welche Strecken man zerstören müsste, um den Nachschub am schnellsten Abzuschneiden. Die Untersuchung bietet uns sogleich einen intuitiven Begriff des \emph{maximalen Flusses} und des \emph{minimalen Schnitts}. Der \emph{maximale Fluss} ist die maximale Anzahl an Einheiten, die von der Quelle zur Senke transportiert werden kann, während der \emph{minimale Schnitt} den \enquote{Bottleneck} des Netzwerkes darstellt, der den Fluss von der Quelle zur Senke minimiert. Wie wir später formal zeigen werden sind diese beiden Werte in einem Flussnetzwerk äquivalent.

Zunächst betrachten wir Flussnetzwerke auf einer formalen Ebene, dann widmen wir uns kurz und skizziert einigen der wichtigsten Algorithmen zum berechnen von \emph{maximalen Flüssen} und danach beweisen wir die Äquivalenz von diesen mit \emph{minimalen Schnitten}. Diese \emph{minimalen Schnitte} brauchen wir später zum Auflösen von \emph{Inkonsistenzen} im \emph{Inzidenzgraphen}.

\begin{definition}[Flussnetzwerk]
	Ein \emph{Flussnetzwerk} ist ein gerichteter Graph $G = (V,E)$ bei dem jeder Kante $(u,v) \in E$ eine nicht-negative Kapazität $c(u,v) \geq 0$ zugeordnet ist und bei dem es zwei ausgezeichnete Knoten $s, t \in V$ gibt, die wir \emph{Quelle} und \emph{Senke} nennen.  
\end{definition}

Der Einfachheit halber nehmen wir im Folgenden an, dass jeder Knoten von $G$ auf einem Pfad von $s$ nach $t$ liegt. Sollte es doch solche Knoten geben, dann wären sie ohnehin nicht relevant, weil kein \emph{Fluss} durch sie von der \emph{Quelle} zur \emph{Senke} geschickt werden kann.

\begin{definition}[Fluss]
	Sei $G = (V,E)$ mit Kapazitätsfunktion $c$ ein \emph{Flussnetzwerk}. Dann ist definieren wir einen \emph{Fluss} als eine Funktion $f: V \times V \rightarrow \mathbb{R}$, die die folgenden beiden Eigenschaften erfüllt:
	\begin{itemize}[leftmargin=12em]
		\item[\textbf{Kapazitätsbeschränkung:}] Für alle Knoten $u,v \in V$ gelte $0 \leq f(u,v) \leq c(u,v)$.
		\item[\textbf{Flusserhaltung:}] Für alle Knoten $u \in V\setminus\{s,t\}$ gelte \\ 
			$\sum_{v \in V}{f(v,u)} = \sum_{v \in V}{f(u,v)}$.
	\end{itemize} 
	Für Knoten $u,v\in V$, die nicht durch eine Kante verbunden sind ($(u,v)\notin E$), setzen wir den Fluss auf 0: $f(u,v) = 0$.
\end{definition}

Ein \emph{Fluss} ist also eine Zuordnung von Werten an die Kanten unseres \emph{Flussnetzwerkes}, sodass keine Kapazität verletzt wird und in jeden Knoten soviel hinein wie hinaus fließt. Wir nennen auch den Wert $f(u,v)$ \emph{Fluss} zwischen den beiden Knoten $u$ und $v$.


Hier ein Beispiel für ein einfaches \emph{Flussnetzwerk} mit sechs Knoten:
\begin{center}
	\begin{tikzpicture}[
	mycircle/.style={
		circle,
		draw=black,
		fill=gray,
		fill opacity = 0.3,
		text opacity=1,
		inner sep=0pt,
		minimum size=20pt,
		font=\footnotesize},
	myarrow/.style={-Stealth},
	node distance=0.6cm and 1.2cm
	]
	\node[mycircle] (c1) {$s$};
	\node[mycircle,below right=of c1] (c2) {$v_2$};
	\node[mycircle,right=of c2] (c3) {$v_4$};
	\node[mycircle,above right=of c1] (c4) {$v_1$};
	\node[mycircle,right=of c4] (c5) {$v_3$};
	\node[mycircle,below right=of c5] (c6) {$t$};
	
	\foreach \i/\j/\txt/\p in {% start node/end node/text/position
		c1/c2/8/below,
		c1/c4/11/above,
		c2/c3/11/below,
		c3/c6/4/below,
		c4/c5/12/above,
		c5/c6/15/above,
		c5/c2/4/below,
		c3/c5/7/below,
		c2.70/c4.290/1/below}
	\draw [myarrow] (\i) -- node[sloped,font=\footnotesize,\p] {\txt} (\j);
	
	
	% draw this outside loop to get proper orientation of 10
	\draw [myarrow] (c4.250) -- node[sloped,font=\small,above,rotate=180] {10} (c2.110);
	\end{tikzpicture}
\end{center}

Viele Autoren setzen voraus, dass es keine Kanten in die \emph{Senke} und keine aus der \emph{Quelle} gibt. Diese Beschränkung ist zwar hilfreich fürs Verständnis, wird aber grundsätzlich nicht benötigt. Man kann sich leicht überlegen, dass selbst wenn es solche Kanten gibt, ihr \emph{Fluss} 0 sein muss. Bei den \emph{Flussnetzwerken}, die wir für unseren Algorithmus benutzen, wird es diese Kanten geben, weshalb wir auf die Beschränkung verzichten. 

\begin{definition}[Wert eines Flusses]
	Der \emph{Wert} eines \emph{Flusses} ist definiert als
	\begin{equation}
		|f| = \sum_{v \in V}{f(s,v)} - \sum_{v \in V}{f(v,s)}.
	\end{equation} 
\end{definition}

Eine andere Einschränkung, die man in vielen Definitionen sieht, ist der Verzicht auf antiparallele Kanten. Das heißt, dass es gleichzeitig Kanten $(u,v), (v,u) \in E$ gibt. Manche Implementierungen für Algorithmen, die den maximalen Fluss berechnen, sind so programmiert, dass sie vom Benutzer eine Menge von Gegenkanten mit \emph{Fluss} 0 erwarten, die intern benötigt werden. Grundsätzlich sind antiparallele Kanten aber unproblematisch, weil man jeden \emph{Fluss} mit positiven Werten auf antiparallelen Kanten in einen \emph{Fluss} umwandeln kann, bei dem zwischen zwei Knoten höchstens eine Kante einen positiven \emph{Fluss} hat. Sei beispielsweise $f(u,v) = x, f(v,u) = y$ mit $x \not 0 \not y$. Dann kann man einen \emph{Fluss} mit dem selben \emph{Wert} definieren, wenn man den \emph{Fluss} auf den antiparallelen Kanten wie folgt definiert: $f(u,v) = x - \min\{x,y\}, f(v,u) = y - \min\{x,y\}$. Alternativ lässt sich jedes \emph{Flussnetzwerk} mit antiparallelen Kanten auch in ein äquivalentes übertragen, das keine solchen Kanten enthält.

\begin{center}
\begin{tikzpicture}[
	mycircle/.style={
		circle,
		draw=black,
		fill=gray,
		fill opacity = 0.3,
		text opacity=1,
		inner sep=0pt,
		minimum size=20pt,
		font=\footnotesize},
	myarrow/.style={-Stealth},
	node distance=0.6cm and 1.2cm
	]
	\node[mycircle] at (0,0) (c1) {$u$};
	\node[mycircle] at (0,-2) (c2) {$v$};
	
	\node[mycircle] at (5,0) (c3) {$u$};
	\node[mycircle] at (3,-1) (c4) {$c$};
	\node[mycircle] at (7,-1) (c5) {$c'$};
	\node[mycircle] at (5,-2) (c6) {$v$};

	
	\foreach \i/\j/\txt/\p in {% start node/end node/text/position
		c1.255/c2.105/2/below,
		c2.75/c1.285/5/below,
		c3/c4/2/below,
		c4/c6/2/below,
		c6/c5/5/below,
		c5/c3/5/below}
	\draw [myarrow] (\i) -- node[sloped,font=\footnotesize,\p] {\txt} (\j);
\end{tikzpicture}
\end{center}

\subsection{Wichtige Algorithmen}

Es gibt eine Vielzahl von Algorithmen zur Berechnung eines maximalen Flusses auf einem Flussnetzwerk. Die wichtigsten und bekanntesten Vertreter möchte ich kurz vorstellen. 

Der erste Algorithmus, der speziell zum berechnen des \emph{maximalen Flusses} entwickelt wurde war der Algorithmus von Ford-Fulkerson \citep{gt14}. Dieser verwendete einen sogenannten \emph{Restwertgraphen}, der für jede Kante im ursprünglichen \emph{Flussnetzwerk} eine Vorwärts- und eine Rückwärtskante enthält. Die Vorwärtskante hat als Kapazität die der Kante im ursprünglichen Graph minus den \emph{Fluss} im \emph{Flussnetzwerk}. Die der Rückwärtskante im \emph{Restwertgraphen} ist genau die Kapazität des \emph{Flusses} der zugehörigen Kante im \emph{Flussnetzwerk}. Für einen Pfad von der \emph{Quelle} zur \emph{Senke} im \emph{Restwertgraph} kann man den \emph{Fluss} im \emph{Flussnetzwerk} um die kleinste Kapazität auf diesem Pfad erhöhen, ohne die \emph{Kapazitätsbeschränkung} zu verletzen. Diese Pfade nennt man \emph{augmentierend} und nach jeder Aktualisierung des Flusses, müssen auch die Kapazitäten im \emph{Restwertgraphen} angepasst werden. Der Algorithmus von Ford-Fulkerson besucht solange \emph{augmentierende} Pfade, bis es keine solchen mehr gibt. Ist das der Fall, dann kann der \emph{Fluss} nicht mehr vergrößert werden und der \emph{maximale Fluss} wurde berechnet. Der Algorithmus von Ford-Fulkerson hat für einen zusammenhängenden Graphen mit ganzzahligen Kapazitäten eine pseudopolynomielle Laufzeit von $\oh(|E|\cdot U)$, wobei $U$ die Summe der Kapazitäten der ausgehenden Kanten von der \emph{Quelle} ist.

Oft wird Ford-Fulkerson eher als eine \enquote{Methode} statt als \emph{Algorithmus} bezeichnet, weil die Reihenfolge nach der \emph{augmentierende} Pfade gewählt werden nicht spezifiziert ist. In der Folge wurde der Algorithmus von Ford-Fulkerson an mehreren Stellen verbessert. Edmond-Karps benutzt eine Breitensuche, um den Pfad mit den wenigsten Knoten von der \emph{Quelle} zur \emph{Senke} zu finden. Dadurch lässt sich die Laufzeit auf $\oh(|V|\cdot |E|^2)$ verbessern \citep{clrs09}. Auch der Algorithmus von Dinic \footnote{Eigentlich entwickelt vom sowjetischen Forscher Yefim A. Dinitz. Die verbreitete Schreibweise beruht auf einem Übersetzungsfehler. Später wanderte Dinitz nach Israel aus \citep{d06}.} sucht nach kürzesten Pfaden im \emph{Restwertgraphen}. Zusätzlich wird das Konzept von sogenannten \emph{blockierenden Flüssen} benutzt, bei denen jeder $s-t$-Pfad mindestens eine Kante enthält, deren Kapazität komplett ausgereizt ist. Der Algorithmus von Dinic benötigt nur $\oh(|V|^2\cdot |E|)$ Rechenschritte \citep{d06}.

Die nächste Klasse von Algorithmen waren die sogenannten \emph{Push-Relabel-Algorithmen}. Diese benutzen bei ihren Schritten statt \emph{Flüssen} nur sogenannten \emph{Präflüsse} bei denen zwar die \emph{Kapazitätsbeschränkung} gilt, die \emph{Flusserhaltung} aber nicht. Das heißt, dass jeder Knoten einen positiven \emph{Überfluss} haben kann, wenn mehr in ihn hinein als hinaus fließt. Jeder Knoten hat einen Index, der seine Höhe im Netz angibt, wobei \emph{Fluss} immer nur von oben nach unten geleitet werden kann. Wie der Name schon sagt, sind die zwei Grundschritte bei allen \emph{Push-Relabel-Algorithmen} die Methoden \emph{Push} und \emph{Relabel}. Bei \emph{Push} verschieben wir den \emph{Überfluss} eines Knotens zu einem tiefer gelegenen Nachbarn. Bei \emph{Relabel} wird hingegen die Höhe eines Knotens vergrößert, wenn er noch \emph{Überfluss} hat, aber alle benachbarten Knoten einen größeren Index haben. Die \emph{Push-} und \emph{Relabelmethoden} werden solange angewendet, bis es keine Knoten mehr gibt auf die sie anwendbar sind \citep{clrs09}. Je nachdem wie der nächste zu bearbeitende Knoten ausgewählt wird, kann die Laufzeit stark variieren. Ein generischer Algorithmus, der die Knoten mehr oder minder zufällig auswählt, läuft in $\oh(|V|^2\cdot |E|)$. Mit einer FIFO-Warteschlange in dem jeder bearbeitete Knoten, der noch \emph{Überfluss} hat, wieder ans Ende eingereiht wird, lässt sich die Laufzeit auf $\oh(|V|^3)$ verbessern \citep{gt88}. Der in der Praxis verbreitetste Ansatz benutzt die \enquote{Highest-Label}-Regel, bei der alle Knoten in Töpfen nach ihrem Index eingeordnet sind und jeweils ein Knoten mit maximalem Index ausgewählt wird. Hier verbessert sich die Laufzeit auf $\oh(|V|^2\cdot \sqrt{|E|})$ Rechenschritte im schlimmsten Fall \citep{akmo97}.

Weiter verbessern ließ sich die asymptotische Laufzeit durch die Verwendung von \emph{dynamischen Bäumen}. \emph{Dynamische Bäume} sind eine Datenstruktur, die die nicht-saturierten Teile von \emph{augmentierenden} Pfaden speichern. Mit Operationen auf dieser Datenstruktur ist es möglich die Laufzeit zu senken. Beschränkt man beispielsweise die maximale Baumgröße und benutzt eine weitere Datenstruktur, dann lässt sich der maximale Fluss mit Hilfe von \emph{blockierenden Flüssen} in $\oh(|V|\cdot |E|\cdot \log(|V|^2/|E|))$ berechnen \citep{gt14}. In der Praxis sind diese Algorithmen aber nicht von Relevanz, weil die verwendete Baumstruktur einen großen konstanten Vorfaktor benötigt. Einer der schnellsten bekannten Algorithmen ist der von Orlin. Er verwendet eine Kombination verschiedener Techniken in unterschiedlichen Situationen und kommt insgesamt auf eine Laufzeit von $\oh(|V|\cdot |E|)$ \citep{gt14}.
 
\subsection{Schnitte und der \emph{Max-Flow-Min-Cut-Satz}}

\begin{definition}[Schnitt]
	Sei $G = (V,E)$ mit \emph{Senke} und \emph{Quelle} $s, t \in V$ und einer Kapazitätsfunktion $c$. Dann ist ein \emph{Schnitt} von $G$ eine Partitionierung von $V$ in zwei Mengen $A$ und $B$, sodass $s \in A$ und $t \in B$ gelten.
	
	Die Kapazität $c_{A,B}$ dieses \emph{Schnitts} ist definiert als die Summe der Kapazitäten aller Kanten von $A$ nach $B$:
	\begin{equation}
		c_{A,B} \coloneqq \sum_{(v,w) \in E \cap (A \times B)}{c(v,w)}
	\end{equation}
\end{definition}

Betrachten wir ein Beispiel für einen \emph{Schnitt} auf einem \emph{Flussnetzwerk}. Alle Knoten in $A$ wurden grün und alle in $B$ rot markiert. Kanten zwischen den beiden Mengen wurden gepunktet dargestellt.

\vspace{-8pt}
\begin{center}
	\begin{tikzpicture}[
	mygreennode/.style={
		circle,
		draw=black,
		fill=green,
		fill opacity = 0.3,
		text opacity=1,
		inner sep=0pt,
		minimum size=20pt,
		font=\footnotesize},
	myrednode/.style={
		circle,
		draw=black,
		fill=red,
		fill opacity = 0.3,
		text opacity=1,
		inner sep=0pt,
		minimum size=20pt,
		font=\footnotesize},
	myarrow/.style={-Stealth},
	node distance=0.6cm and 1.2cm
	]
	\node[mygreennode] (c1) {$s$};
	\node[myrednode,below right=of c1] (c2) {$v_2$};
	\node[mygreennode,right=of c2] (c3) {$v_4$};
	\node[mygreennode,above right=of c1] (c4) {$v_1$};
	\node[mygreennode,right=of c4] (c5) {$v_3$};
	\node[myrednode,below right=of c5] (c6) {$t$};
	
	\foreach \i/\j/\txt/\p in {% start node/end node/text/position
		c1/c4/11/above,
		c4/c5/12/above,
		c5/c6/15/above,
		c3/c5/7/below}
	\draw [myarrow] (\i) -- node[sloped,font=\footnotesize,\p] {\txt} (\j);
	
	\foreach \i/\j/\txt/\p in {% start node/end node/text/position
		c1/c2/8/below,
		c2/c3/11/below,
		c3/c6/4/below,
		c5/c2/4/below,
		c2.70/c4.290/1/below}
	\draw [myarrow, dotted] (\i) -- node[sloped,font=\footnotesize,\p] {\txt} (\j);
	
	% draw this outside loop to get proper orientation of 10
	\draw [myarrow, dotted] (c4.250) -- node[sloped,font=\small,above,rotate=180] {10} (c2.110);
	\end{tikzpicture}
\end{center}
\vspace{-8pt}

Es gilt $A = \{s, v_1, v_3, v_4\}$ und $B = \{v_2, t\}$. Zur Kapazität des \emph{Schnitts} tragen alle Kanten bei, die von Knoten aus $A$ zu solchen aus $B$ verlaufen. Die Kante $(v_3,v_2)$ also schon, die Kante $(v_2, v_4)$ aber nicht. Insgesamt beträgt die Kapazität des \emph{Schnitts} $c(s,v_2) + c(v_1, v_2) + c(v_3,v_2) + c(v_3,t) + c(v_4,t) = 8 + 10 + 4 + 15 + 4 = 41$.  

\begin{definition}[Minimaler Schnitt]
	Als \emph{minimalen Schnitt} bezeichnet man den oder einen Schnitt mit minimaler Kapazität. Es kann mehrere solcher \emph{minimaler Schnitte} geben. 
\end{definition}

Dem geneigten Leser mag die Asymmetrie zwischen \emph{Flüssen}, die maximal sein können und \emph{Schnitten}, die minimal sein können, aufgefallen sein. Den Zusammenhang der beiden Konzepte möchte ich im Folgenden kurz beschreiben. Dazu werden wir zwei formale Aussagen betrachten, die hier aber nicht bewiesen werden. Diese sind zwar für unseren Algorithmus von Bedeutung, aber für einen formalen Beweis bräuchten wir einen größeren theoretischen Hintergrund, der obgleich furchtbar spannend, hier den Rahmen sprengte und wenig zielführend wäre.

\begin{lemma}
	Der \emph{Wert} eines beliebigen \emph{Flusses} über einem \emph{Flussnetzwerk} ist beschränkt durch einen beliebigen Schnitt.
\end{lemma}

\begin{beweis}
	Siehe \cite{clrs09}.
\end{beweis}

\begin{satz}[Max-Flow-Min-Cut-Satz]
	Für ein \emph{Flussnetzwerk} $G = (V,E)$ mit \emph{Senke} und \emph{Quelle} $s, t \in V$ entspricht die minimale Kapazität aller \emph{Schnitte} auf $G$ dem maximalen \emph{Wert} aller Flüsse von $s$ nach $t$. 
\end{satz} 

\begin{beweis}
	Wenn wir an den Algorithmus von Ford-Fulkerson denken, dann erinnern wir uns, dass dieser solange \emph{augmentierende} Pfade mit positiver \emph{Restwertkapazität} von der \emph{Quelle} zur \emph{Senke} gesucht hat, bis es keinen solchen mehr gab. Sobald das der Fall war, hatte man den \emph{maximalen Fluss} berechnet. Das bedeutet gleichzeitig, dass man den \emph{Bottleneck} zwischen $s$ und $t$ gefunden hat. Genau dieser trennt die Knoten von $G$ in zwei Hälften und ist unser \emph{minimaler Schnitt}. Für einen formalen Beweis neben dieser Skizze siehe auch hier \cite{clrs09}.
\end{beweis} 

Die Erkenntnis, die man aus diesem Satz zieht, ist, dass man Algorithmen zur Berechnung des \emph{maximalen Flusses} benutzen kann, um den \emph{minimalen Schnitt} eines \emph{Flussnetzwerkes} zu bestimmen. Das werden wir im nächsten Abschnitt ausnutzen, um \emph{Inkonsistenzen} zwischen unseren Sequenzen aufzulösen.

\section{Inzidenzgraphen und das Auflösen von Inkonsistenzen mit Hilfe von Flussnetzwerken}

\subsection{Komplexität}

Möglicherweise bessere Laufzeit, wenn alle Kanten Kapazität 1 haben (s. Karzanov/Even); für Push-Relabel nichts gefunden.

$\oh(n^4\cdot L^{7/2})$

\section{Sukzessorgraphen und der Algorithmus von Pitschi}
$\oh(n^3\cdot L)$ für n-faches Topological Sort auf einem Knoten mit $\oh(n^2\cdot L)$-vielen Kanten im schlimmsten Fall.

\section{Ankerpunkte}

\section{Gesamtkomplexität}
$\oh(n^4\cdot L^{7/2})$ 