\chapter{Ein Min-Cut-Ansatz für das Konsistenzproblem}
\label{ch:min-cut}
Da es, wie im letzten Kapitel gezeigt, auf manchen Sequenzfamilien durch die gierige Heuristik von DIALIGN zu suboptimalen \emph{Scores} und \emph{Alignments} kommt, werden wir jetzt einen verbesserten graphtheoretischen Ansatz von \cite{cpm10} betrachten.

Dazu benötigen wir zwei verschiedene Graphen: zum einen den \emph{Inzidenzgraphen}, bei dem alle \emph{Stellen} Knoten sind und ihre \emph{Anker} Zusammenhangskomponenten. Der zweite ist der \emph{Sukzessionsgraph}, der die Zusammenhangskomponenten unseres \emph{Inzidenzgraphen} als Knoten und die natürliche Ordnung auf den Sequenzen als Kanten benutzt. Man kann sich vorstellen, dass genau dann eine Kante zwischen zwei  Diese beiden Datenstrukturen werden wir benutzen, um \emph{Inkonsistenzen} aufzulösen. Wenn wir uns an die Definition von \emph{Konsistenz} aus dem letzten Kapitel erinnern, dann stellen wir fest, dass es zwei Arten von ihr gibt: zum einen implizite, transitive Mehrfachzuweisungen bei denen einer \emph{Stelle} einer Sequenze mehrere \emph{Stellen} einer anderen Sequenz zugeordnet sind und zum anderen überkreuzte Zuweisungen.

Als Ausgangspunkt starten wir wieder mit unseren paarweisen \emph{Alignments} aus DIALIGN. \emph{Überlappgewichte} brauchen wir in unserem Fall nicht. Dann konstruieren wir mit Hilfe dieser Zuweisungen unseren \emph{Inzidenzgraphen} und benutzen einen Algorithmus zur Berechnung des minimalen Schnitts (\enquote{min-cut}) auf den Zusammenhangskomponenten, um alle \emph{Inkonsistenzen} aufgrund von transitiven Mehrfachzuweisungen aufzulösen. Die so entstehenden Zusammenhangskomponenten benutzen wir, um einen \emph{Sukzessionsgraphen} aufzubauen. Dank eines Algorithmus von \cite{pdc10} können wir mit diesem Überkreuzungen aus unserer Relation löschen. Alle dieser Konzepte werden wir im Laufe dieses Kapitel formal definieren, genauer analysieren und die Korrektheit der Aussagen beweisen.
 
\section{Flussnetzwerke}

\subsection{Einführung}

Stell dir eine Produktionsstätte vor, wo ein bestimmtes Produkt hergestellt wird und eine Verwendungsstätte bei der das Produkt benötigt wird. Zwischen dem Ort der Produktion und dem Ort der Verwendung gibt es ein Schienennetz über das die Produkte geliefert werden können. Über jeden Abschnitt der Schienen kann aber nur eine bestimmte Anzahl an Waggons geleitet werden. So mag es weniger stark befestigte Strecken geben, wo nur kleinere oder leichtere Züge fahren können und andere gut ausgebaute mit mehreren Gleisen nebeneinander. Die Anzahl an Einheiten, die über einen Abschnitt geleitet werden können, nennt man Kapazität, während Produktionsstätte und Zielort Quelle und Senke heißen. Ziel ist es zu bestimmen wie viele Produktionseinheiten über dieses Netz von der Fabrik zum Verbraucher geliefert werden können.

Neben diesem Problem kann man mit Flussnetzwerken noch viele andere Anwendungen modellieren, zum Beispiel eine chemische Produktionsstraße mit vielen Rohren, die jeweils nur den Durchfluss einer bestimmten Menge erlauben. Oder Stromnetze mit Kraftwerken und Endverbrauchern zwischen denen es Stromleitungen gibt, die jeweils höchstens einen bestimmten Stromfluss erlauben. Für eine Hochspannungsleitung mag dieser sehr hoch und für die Leitungen, die direkt zu den Häusern gehen, sehr klein sein. Stromnetze bieten uns auch eine direkte Analogie, wie mit den Knotenpunkten zwischen den Verbindungen umzugehen ist. An jedem Knoten, außer der Quelle und Senke, wird genauso viel Fluss hinein- wie wieder hinausgeleitet. Es wird nichts gespeichert oder geht verloren. Dieses Konzept, das man \emph{Flusserhaltung} nennt, funktioniert genau wie das erste \emph{Kirchhoffsche Gesetz} bei Strömen \citep{clrs09}. 

Ursprünglich stammt das Konzept des Flussnetzwerkes aus dem Kalten Krieg und der militärischen Forschung. Es wurde das Schienennetzwerk des Sowjetunion in Osteuropa untersucht, um herauszufinden wie viel Material die UdSSR im Kriegsfall aus dem russischen Kerngebiet nach Mitteleuropa transportieren könnte und welche Strecken man zerstören müsste, um den Nachschub am schnellsten Abzuschneiden. Die Untersuchung bietet uns sogleich einen intuitiven Begriff des \emph{maximalen Flusses} und des \emph{minimalen Schnitts}. Der \emph{maximale Fluss} ist die maximale Anzahl an Einheiten, die von der Quelle zur Senke transportiert werden kann, während der \emph{minimale Schnitt} den \enquote{Bottleneck} des Netzwerkes darstellt, der den Fluss von der Quelle zur Senke minimiert. Wie wir später formal zeigen werden sind diese beiden Werte in einem Flussnetzwerk äquivalent.

Zunächst betrachten wir Flussnetzwerke auf einer formalen Ebene, dann widmen wir uns kurz und skizziert einigen der wichtigsten Algorithmen zum berechnen von \emph{maximalen Flüssen} und danach beweisen wir die Äquivalenz von diesen mit \emph{minimalen Schnitten}. Diese \emph{minimalen Schnitte} brauchen wir später zum Auflösen von \emph{Inkonsistenzen} im \emph{Inzidenzgraphen}.

\begin{definition}[Flussnetzwerk]
	Ein \emph{Flussnetzwerk} ist ein gerichteter Graph $G = (V,E)$ bei dem jeder Kante $(u,v) \in E$ eine nicht-negative Kapazität $c(u,v) \geq 0$ zugeordnet ist und bei dem es zwei ausgezeichnete Knoten $s, t \in V$ gibt, die wir \emph{Quelle} und \emph{Senke} nennen.  
\end{definition}

Der Einfachheit halber nehmen wir im Folgenden an, dass jeder Knoten von $G$ auf einem Pfad von $s$ nach $t$ liegt. Sollte es doch solche Knoten geben, dann wären sie ohnehin nicht relevant, weil kein \emph{Fluss} durch sie von der \emph{Quelle} zur \emph{Senke} geschickt werden kann.

\begin{definition}[Fluss]
	Sei $G = (V,E)$ mit Kapazitätsfunktion $c$ ein \emph{Flussnetzwerk}. Dann ist definieren wir einen \emph{Fluss} als eine Funktion $f: V \times V \rightarrow \mathbb{R}$, die die folgenden beiden Eigenschaften erfüllt:
	\begin{itemize}[leftmargin=12em]
		\item[\textbf{Kapazitätsbeschränkung:}] Für alle Knoten $u,v \in V$ gelte $0 \leq f(u,v) \leq c(u,v)$.
		\item[\textbf{Flusserhaltung:}] Für alle Knoten $u \in V\setminus\{s,t\}$ gelte \\ 
			$\sum_{v \in V}{f(v,u)} = \sum_{v \in V}{f(u,v)}$.
	\end{itemize} 
	Für Knoten $u,v\in V$, die nicht durch eine Kante verbunden sind ($(u,v)\notin E$), setzen wir den Fluss auf 0: $f(u,v) = 0$.
\end{definition}

Ein \emph{Fluss} ist also eine Zuordnung von Werten an die Kanten unseres \emph{Flussnetzwerkes}, sodass keine Kapazität verletzt wird und in jeden Knoten soviel hinein wie hinaus fließt. Wir nennen auch den Wert $f(u,v)$ \emph{Fluss} zwischen den beiden Knoten $u$ und $v$.


Hier ein Beispiel für ein einfaches \emph{Flussnetzwerk} mit sechs Knoten:
\begin{center}
	\begin{tikzpicture}[
	mycircle/.style={
		circle,
		draw=black,
		fill=gray,
		fill opacity = 0.3,
		text opacity=1,
		inner sep=0pt,
		minimum size=20pt,
		font=\footnotesize},
	myarrow/.style={-Stealth},
	node distance=0.6cm and 1.2cm
	]
	\node[mycircle] (c1) {$s$};
	\node[mycircle,below right=of c1] (c2) {$v_2$};
	\node[mycircle,right=of c2] (c3) {$v_4$};
	\node[mycircle,above right=of c1] (c4) {$v_1$};
	\node[mycircle,right=of c4] (c5) {$v_3$};
	\node[mycircle,below right=of c5] (c6) {$t$};
	
	\foreach \i/\j/\txt/\p in {% start node/end node/text/position
		c1/c2/8/below,
		c1/c4/11/above,
		c2/c3/11/below,
		c3/c6/4/below,
		c4/c5/12/above,
		c5/c6/15/above,
		c5/c2/4/below,
		c3/c5/7/below,
		c2.70/c4.290/1/below}
	\draw [myarrow] (\i) -- node[sloped,font=\footnotesize,\p] {\txt} (\j);
	
	
	% draw this outside loop to get proper orientation of 10
	\draw [myarrow] (c4.250) -- node[sloped,font=\small,above,rotate=180] {10} (c2.110);
	\end{tikzpicture}
\end{center}

Viele Autoren setzen voraus, dass es keine Kanten in die \emph{Senke} und keine aus der \emph{Quelle} gibt. Diese Beschränkung ist zwar hilfreich fürs Verständnis, wird aber grundsätzlich nicht benötigt. Man kann sich leicht überlegen, dass selbst wenn es solche Kanten gibt, ihr \emph{Fluss} 0 sein muss. Bei den \emph{Flussnetzwerken}, die wir für unseren Algorithmus benutzen, wird es diese Kanten geben, weshalb wir auf die Beschränkung verzichten. 

\begin{definition}[Wert eines Flusses]
	Der \emph{Wert} eines \emph{Flusses} ist definiert als
	\begin{equation}
		|f| = \sum_{v \in V}{f(s,v)} - \sum_{v \in V}{f(v,s)}.
	\end{equation} 
\end{definition}

Eine andere Einschränkung, die man in vielen Definitionen sieht, ist der Verzicht auf antiparallele Kanten. Das heißt, dass es gleichzeitig Kanten $(u,v), (v,u) \in E$ gibt. Manche Implementierungen für Algorithmen, die den maximalen Fluss berechnen, sind so programmiert, dass sie vom Benutzer eine Menge von Gegenkanten mit \emph{Fluss} 0 erwarten, die intern benötigt werden. Grundsätzlich sind antiparallele Kanten aber unproblematisch, weil man jeden \emph{Fluss} mit positiven Werten auf antiparallelen Kanten in einen \emph{Fluss} umwandeln kann, bei dem zwischen zwei Knoten höchstens eine Kante einen positiven \emph{Fluss} hat. Sei beispielsweise $f(u,v) = x, f(v,u) = y$ mit $x \not 0 \not y$. Dann kann man einen \emph{Fluss} mit dem selben \emph{Wert} definieren, wenn man den \emph{Fluss} auf den antiparallelen Kanten wie folgt definiert: $f(u,v) = x - \min\{x,y\}, f(v,u) = y - \min\{x,y\}$. Alternativ lässt sich jedes \emph{Flussnetzwerk} mit antiparallelen Kanten auch in ein äquivalentes übertragen, das keine solchen Kanten enthält.

\begin{center}
\begin{tikzpicture}[
	mycircle/.style={
		circle,
		draw=black,
		fill=gray,
		fill opacity = 0.3,
		text opacity=1,
		inner sep=0pt,
		minimum size=20pt,
		font=\footnotesize},
	myarrow/.style={-Stealth},
	node distance=0.6cm and 1.2cm
	]
	\node[mycircle] at (0,0) (c1) {$u$};
	\node[mycircle] at (0,-2) (c2) {$v$};
	
	\node[mycircle] at (5,0) (c3) {$u$};
	\node[mycircle] at (3,-1) (c4) {$c$};
	\node[mycircle] at (7,-1) (c5) {$c'$};
	\node[mycircle] at (5,-2) (c6) {$v$};

	
	\foreach \i/\j/\txt/\p in {% start node/end node/text/position
		c1.255/c2.105/2/below,
		c2.75/c1.285/5/below,
		c3/c4/2/below,
		c4/c6/2/below,
		c6/c5/5/below,
		c5/c3/5/below}
	\draw [myarrow] (\i) -- node[sloped,font=\footnotesize,\p] {\txt} (\j);
\end{tikzpicture}
\end{center}

\subsection{Wichtige Algorithmen}

Es gibt eine Vielzahl von Algorithmen zur Berechnung eines maximalen Flusses auf einem Flussnetzwerk. Die wichtigsten und bekanntesten Vertreter möchte ich kurz vorstellen. 

Der erste Algorithmus, der speziell zum berechnen des \emph{maximalen Flusses} entwickelt wurde war der Algorithmus von Ford-Fulkerson \citep{gt14}. Dieser verwendete einen sogenannten \emph{Restwertgraphen}, der für jede Kante im ursprünglichen \emph{Flussnetzwerk} eine Vorwärts- und eine Rückwärtskante enthält. Die Vorwärtskante hat als Kapazität die der Kante im ursprünglichen Graph minus den \emph{Fluss} im \emph{Flussnetzwerk}. Die der Rückwärtskante im \emph{Restwertgraphen} ist genau die Kapazität des \emph{Flusses} der zugehörigen Kante im \emph{Flussnetzwerk}. Für einen Pfad von der \emph{Quelle} zur \emph{Senke} im \emph{Restwertgraph} kann man den \emph{Fluss} im \emph{Flussnetzwerk} um die kleinste Kapazität auf diesem Pfad erhöhen, ohne die \emph{Kapazitätsbeschränkung} zu verletzen. Diese Pfade nennt man \emph{augmentierend} und nach jeder Aktualisierung des Flusses, müssen auch die Kapazitäten im \emph{Restwertgraphen} angepasst werden. Der Algorithmus von Ford-Fulkerson besucht solange \emph{augmentierende} Pfade, bis es keine solchen mehr gibt. Ist das der Fall, dann kann der \emph{Fluss} nicht mehr vergrößert werden und der \emph{maximale Fluss} wurde berechnet. Der Algorithmus von Ford-Fulkerson hat für einen zusammenhängenden Graphen mit ganzzahligen Kapazitäten eine pseudopolynomielle Laufzeit von $\oh(|E|\cdot U)$, wobei $U$ die Summe der Kapazitäten der ausgehenden Kanten von der \emph{Quelle} ist.

Oft wird Ford-Fulkerson eher als eine \enquote{Methode} statt als \emph{Algorithmus} bezeichnet, weil die Reihenfolge nach der \emph{augmentierende} Pfade gewählt werden nicht spezifiziert ist. In der Folge wurde der Algorithmus von Ford-Fulkerson an mehreren Stellen verbessert. Edmond-Karps benutzt eine Breitensuche, um den Pfad mit den wenigsten Knoten von der \emph{Quelle} zur \emph{Senke} zu finden. Dadurch lässt sich die Laufzeit auf $\oh(|V|\cdot |E|^2)$ verbessern \citep{clrs09}. Auch der Algorithmus von Dinic \footnote{Eigentlich entwickelt vom sowjetischen Forscher Yefim A. Dinitz. Die verbreitete Schreibweise beruht auf einem Übersetzungsfehler. Später wanderte Dinitz nach Israel aus \citep{d06}.} sucht nach kürzesten Pfaden im \emph{Restwertgraphen}. Zusätzlich wird das Konzept von sogenannten \emph{blockierenden Flüssen} benutzt, bei denen jeder $s-t$-Pfad mindestens eine Kante enthält, deren Kapazität komplett ausgereizt ist. Der Algorithmus von Dinic benötigt nur $\oh(|V|^2\cdot |E|)$ Rechenschritte \citep{d06}.

Die nächste Klasse von Algorithmen waren die sogenannten \emph{Push-Relabel-Algorithmen}. Diese benutzen bei ihren Schritten statt \emph{Flüssen} nur sogenannten \emph{Präflüsse} bei denen zwar die \emph{Kapazitätsbeschränkung} gilt, die \emph{Flusserhaltung} aber nicht. Das heißt, dass jeder Knoten einen positiven \emph{Überfluss} haben kann, wenn mehr in ihn hinein als hinaus fließt. Jeder Knoten hat einen Index, der seine Höhe im Netz angibt, wobei \emph{Fluss} immer nur von oben nach unten geleitet werden kann. Wie der Name schon sagt, sind die zwei Grundschritte bei allen \emph{Push-Relabel-Algorithmen} die Methoden \emph{Push} und \emph{Relabel}. Bei \emph{Push} verschieben wir den \emph{Überfluss} eines Knotens zu einem tiefer gelegenen Nachbarn. Bei \emph{Relabel} wird hingegen die Höhe eines Knotens vergrößert, wenn er noch \emph{Überfluss} hat, aber alle benachbarten Knoten einen größeren Index haben. Die \emph{Push-} und \emph{Relabelmethoden} werden solange angewendet, bis es keine Knoten mehr gibt auf die sie anwendbar sind \citep{clrs09}. Je nachdem wie der nächste zu bearbeitende Knoten ausgewählt wird, kann die Laufzeit stark variieren. Ein generischer Algorithmus, der die Knoten mehr oder minder zufällig auswählt, läuft in $\oh(|V|^2\cdot |E|)$. Mit einer FIFO-Warteschlange in dem jeder bearbeitete Knoten, der noch \emph{Überfluss} hat, wieder ans Ende eingereiht wird, lässt sich die Laufzeit auf $\oh(|V|^3)$ verbessern \citep{gt88}. Der in der Praxis verbreitetste Ansatz benutzt die \enquote{Highest-Label}-Regel, bei der alle Knoten in Töpfen nach ihrem Index eingeordnet sind und jeweils ein Knoten mit maximalem Index ausgewählt wird. Hier verbessert sich die Laufzeit auf $\oh(|V|^2\cdot \sqrt{|E|})$ Rechenschritte im schlimmsten Fall \citep{akmo97}.

Weiter verbessern ließ sich die asymptotische Laufzeit durch die Verwendung von \emph{dynamischen Bäumen}. \emph{Dynamische Bäume} sind eine Datenstruktur, die die nicht-saturierten Teile von \emph{augmentierenden} Pfaden speichern. Mit Operationen auf dieser Datenstruktur ist es möglich die Laufzeit zu senken. Beschränkt man beispielsweise die maximale Baumgröße und benutzt eine weitere Datenstruktur, dann lässt sich der maximale Fluss mit Hilfe von \emph{blockierenden Flüssen} in $\oh(|V|\cdot |E|\cdot \log(|V|^2/|E|))$ berechnen \citep{gt14}. In der Praxis sind diese Algorithmen aber nicht von Relevanz, weil die verwendete Baumstruktur einen großen konstanten Vorfaktor benötigt. Einer der schnellsten bekannten Algorithmen ist der von Orlin. Er verwendet eine Kombination verschiedener Techniken in unterschiedlichen Situationen und kommt insgesamt auf eine Laufzeit von $\oh(|V|\cdot |E|)$ \citep{gt14}.
 
\subsection{Schnitte und der \emph{Max-Flow-Min-Cut-Satz}}

\begin{definition}[Schnitt]
	Sei $G = (V,E)$ mit \emph{Senke} und \emph{Quelle} $s, t \in V$ und einer Kapazitätsfunktion $c$. Dann ist ein \emph{Schnitt} von $G$ eine Partitionierung von $V$ in zwei Mengen $A$ und $B$, sodass $s \in A$ und $t \in B$ gelten.
	
	Die Kapazität $c_{A,B}$ dieses \emph{Schnitts} ist definiert als die Summe der Kapazitäten aller Kanten von $A$ nach $B$:
	\begin{equation}
		c_{A,B} \coloneqq \sum_{(v,w) \in E \cap (A \times B)}{c(v,w)}
	\end{equation}
\end{definition}

Betrachten wir ein Beispiel für einen \emph{Schnitt} auf einem \emph{Flussnetzwerk}. Alle Knoten in $A$ wurden grün und alle in $B$ rot markiert. Kanten zwischen den beiden Mengen wurden gepunktet dargestellt.

\vspace{-8pt}
\begin{center}
	\begin{tikzpicture}[
	mygreennode/.style={
		circle,
		draw=black,
		fill=green,
		fill opacity = 0.3,
		text opacity=1,
		inner sep=0pt,
		minimum size=20pt,
		font=\footnotesize},
	myrednode/.style={
		circle,
		draw=black,
		fill=red,
		fill opacity = 0.3,
		text opacity=1,
		inner sep=0pt,
		minimum size=20pt,
		font=\footnotesize},
	myarrow/.style={-Stealth},
	node distance=0.6cm and 1.2cm
	]
	\node[mygreennode] (c1) {$s$};
	\node[myrednode,below right=of c1] (c2) {$v_2$};
	\node[mygreennode,right=of c2] (c3) {$v_4$};
	\node[mygreennode,above right=of c1] (c4) {$v_1$};
	\node[mygreennode,right=of c4] (c5) {$v_3$};
	\node[myrednode,below right=of c5] (c6) {$t$};
	
	\foreach \i/\j/\txt/\p in {% start node/end node/text/position
		c1/c4/11/above,
		c4/c5/12/above,
		c5/c6/15/above,
		c3/c5/7/below}
	\draw [myarrow] (\i) -- node[sloped,font=\footnotesize,\p] {\txt} (\j);
	
	\foreach \i/\j/\txt/\p in {% start node/end node/text/position
		c1/c2/8/below,
		c2/c3/11/below,
		c3/c6/4/below,
		c5/c2/4/below,
		c2.70/c4.290/1/below}
	\draw [myarrow, dotted] (\i) -- node[sloped,font=\footnotesize,\p] {\txt} (\j);
	
	% draw this outside loop to get proper orientation of 10
	\draw [myarrow, dotted] (c4.250) -- node[sloped,font=\small,above,rotate=180] {10} (c2.110);
	\end{tikzpicture}
\end{center}
\vspace{-8pt}

Es gilt $A = \{s, v_1, v_3, v_4\}$ und $B = \{v_2, t\}$. Zur Kapazität des \emph{Schnitts} tragen alle Kanten bei, die von Knoten aus $A$ zu solchen aus $B$ verlaufen. Die Kante $(v_3,v_2)$ also schon, die Kante $(v_2, v_4)$ aber nicht. Insgesamt beträgt die Kapazität des \emph{Schnitts} $c(s,v_2) + c(v_1, v_2) + c(v_3,v_2) + c(v_3,t) + c(v_4,t) = 8 + 10 + 4 + 15 + 4 = 41$.  

\begin{definition}[Minimaler Schnitt]
	Als \emph{minimalen Schnitt} bezeichnet man den oder einen Schnitt mit minimaler Kapazität. Es kann mehrere solcher \emph{minimaler Schnitte} geben. 
\end{definition}

Dem geneigten Leser mag die Asymmetrie zwischen \emph{Flüssen}, die maximal sein können und \emph{Schnitten}, die minimal sein können, aufgefallen sein. Den Zusammenhang der beiden Konzepte möchte ich im Folgenden kurz beschreiben. Dazu werden wir zwei formale Aussagen betrachten, die hier aber nicht bewiesen werden. Diese sind zwar für unseren Algorithmus von Bedeutung, aber für einen formalen Beweis bräuchten wir einen größeren theoretischen Hintergrund, der obgleich furchtbar spannend, hier den Rahmen sprengte und wenig zielführend wäre.

\begin{lemma}
	Der \emph{Wert} eines beliebigen \emph{Flusses} über einem \emph{Flussnetzwerk} ist beschränkt durch einen beliebigen Schnitt.
\end{lemma}

\begin{beweis}
	Siehe \cite{clrs09}.
\end{beweis}

\begin{satz}[Max-Flow-Min-Cut-Satz]
	Für ein \emph{Flussnetzwerk} $G = (V,E)$ mit \emph{Senke} und \emph{Quelle} $s, t \in V$ entspricht die minimale Kapazität aller \emph{Schnitte} auf $G$ dem maximalen \emph{Wert} aller Flüsse von $s$ nach $t$. 
\end{satz} 

\begin{beweis}
	Wenn wir an den Algorithmus von Ford-Fulkerson denken, dann erinnern wir uns, dass dieser solange \emph{augmentierende} Pfade mit positiver \emph{Restwertkapazität} von der \emph{Quelle} zur \emph{Senke} gesucht hat, bis es keinen solchen mehr gab. Sobald das der Fall war, hatte man den \emph{maximalen Fluss} berechnet. Das bedeutet gleichzeitig, dass man den \emph{Bottleneck} zwischen $s$ und $t$ gefunden hat. Genau dieser trennt die Knoten von $G$ in zwei Hälften und ist unser \emph{minimaler Schnitt}. Für einen formalen Beweis neben dieser Skizze siehe auch hier \cite{clrs09}.
\end{beweis} 

Die Erkenntnis, die man aus diesem Satz zieht, ist, dass man Algorithmen zur Berechnung des \emph{maximalen Flusses} benutzen kann, um den \emph{minimalen Schnitt} eines \emph{Flussnetzwerkes} zu bestimmen. Das werden wir im nächsten Abschnitt ausnutzen, um \emph{Inkonsistenzen} zwischen unseren Sequenzen aufzulösen.

\section{Inzidenzgraphen und das Auflösen von Inkonsistenzen mit Hilfe von Flussnetzwerken}

\subsection{Konstruieren des Inzidenzgraphen}

Wir starten wie bei DIALIGN	zunächst mit paarweisen \emph{Alignments}. Unser Ziel ist es Übereinstimmungen zu finden, die in möglichst vielen Sequenzen gleichzeitig vorkommen. Das ist ein vielversprechender Ansatz, weil die gesuchten Motive oft in vielen sich überlappenden \emph{Fragmente} vorkommen, während Überlappungen bei zufälligen Übereinstimmungen unwahrscheinlich sind. Dazu konstruieren wir einen sogenannten \emph{Inzidenzgraphen}, der alle \emph{Stellen} als Knoten enthält, die durch Kanten verbunden sind, falls es ein sie verbindendes \emph{Fragment} gibt. Mit Hilfe eines Algorithmus zur Berechnung des \emph{maximalen Flusses} bestimmen wir \emph{minimale Schnitte}, bis nur noch dichte Zusammenhangskomponenten übrig bleiben, die jeweils höchstens einen Knoten aus jeder Sequenz enthalten \citep{cpm10}.

\begin{definition}[Mehrdeutigkeit und partielle Zuweisungsspalten]
	Es sei eine Menge an Sequenzen $S$ mit \emph{Stellenraum} $\mathcal{S}$ gegeben. Eine Teilmenge $C \subset \mathcal{S}$ nennen wir \emph{mehrdeutig}, wenn es mindestens eine Sequenz $S_i$ gibt, sodass $C \cap S_i$ zwei oder mehr \emph{Stellen} $(i,p), (i,p') \in \mathcal{S}$ enthält. In diesem Fall nennen wir auch $(i,p)$ und $(i,p')$ \emph{mehrdeutig}. Analog nennen wir eine Äquivalenzrelation $\mathcal{R}$ \emph{mehrdeutig}, wenn $\mathcal{R}$ eine \emph{mehrdeutige} Äquivalenzklasse enthält.
	
	Eine \emph{nicht-mehrdeutige} Teilmenge $C \subset \mathcal{S}$ bezeichnen wir als \emph{partielle Zuweisungsspalte}.
\end{definition}

Es lässt sich folgern, dass eine \emph{konsistente} Äquivalenzrelation auch nicht \emph{mehrdeutig} ist, während das Gegenteil im Allgemeinen nicht gilt. Das liegt daran, dass es überkreuzte Zuweisungen geben kann, die aber \emph{partielle Zuweisungsspalten} sind. \emph{Nicht-mehrdeutige} Äquivalenzrelationen bestehen nur aus \emph{partiellen Zuweisungsspalten}.

Im Folgenden sei eine Menge von \emph{Fragmenten} $\mathcal{F} = \{f_1, \dots, f_k\}$ gegeben. Normalerweise werden dies die \emph{Fragmente} unserer paarweisen \emph{Alignments} sein, aber theoretisch kann man auch anders bestimmte benutzen. Wir wollen möglichst wenige Verbindungen aus der durch die \emph{Fragmente} induzierte Relation $\mathcal{R}$ löschen, bis eine \emph{nicht-mehrdeutige} Äquivalenzrelation $\mathcal{R'}$ bleibt.

\begin{definition}[Inzidenzgraph]
	Es sei ein \emph{Stellenraum} $\mathcal{S}$ mit einer Menge von \emph{Fragmenten} $\mathcal{F}$ auf diesem gegeben. Dann bezeichnen wir den ungerichteten Graphen $G_{\mathcal{F}} = (\mathcal{S},E_{\mathcal{F}})$ als \emph{Inzidenzgraphen}. In diesem Graph existiert genau dann eine Kante $(u,v) \in E_{\mathcal{F}}$, wenn die \emph{Stellen} $u$ und $v$ in einem gemeinsamen \emph{Fragment} $f_i \in \mathcal{F}$ vorkommen.
\end{definition} 

Wie man sieht, sind die Zusammenhangskomponenten unseres \emph{Inzidenzgraphen} genau die Äquivalenzklassen der durch $\mathcal{F}$ induzierten Kanten. Weil diese nicht weiter nützlich sind, kann man \emph{Stellen}, die nicht mit anderen verbunden sind, von vornherein ignorieren beziehungsweise sie löschen, wenn sie nach dem Entfernen von Kanten Grad 0 haben. So lässt sich etwas Speicherplatz sparen und die Ergebnisse dieses Algorithmus können direkt für den nächsten Schritt weiterverwendet werden. Um die Positionierung der Zusammenhangskomponenten im \emph{Inzidenzgraph} zu verdeutlichen, lassen wir sie in unseren graphischen Beispiele aber stehen.

\subsection{Beispiel Inzidenzgraph}

Auch hier werden wir wieder unser Beispiel aus dem letzten Kapitel benutzen. Erinnern wir uns zunächst an die \emph{Fragmente} der paarweisen \emph{Alignments}. Die \emph{Überlappgewichte} brauchen wir für diesen Schritt des Algorithmus nicht, weil wir versuchen mit Hilfe von \emph{minimalen Schnitten} ein ähnliches aber besseres Ergebnis zu erreichen.

\begin{tabular}{r|c|c||r|c|c||r|c|c}
	Seq. & \emph{Frag.} & \emph{Ü-Gew.} & Seq. & \emph{Frag.} & \emph{Ü-Gew.} & Seq. & \emph{Frag.} & \emph{Ü-Gew.}\\
	\hline
	2 & \texttt{GTCADCTC} & \multirow{2}{*}{16} & 1 & \texttt{TCTCA} & \multirow{2}{*}{7} & 1 & \texttt{GT} &\multirow{2}{*}{2} \\
	4 & \texttt{GTCADATC} &                     & 3 & \texttt{TATCA} &                     & 2 & \texttt{GT} & \\
	3 & \texttt{TCAD} & \multirow{2}{*}{8} & 1 & \texttt{CTCA} & \multirow{2}{*}{8} & 1 & \texttt{TC} & \multirow{2}{*}{2} \\
	4 & \texttt{TCAD} &                     & 2 & \texttt{CTCA} &                          & 4 & \texttt{TC} & \\
	2 & \texttt{TCAD} & \multirow{2}{*}{8} & 1 & \texttt{DGTC} & \multirow{2}{*}{8} &    &   & \\
	3 & \texttt{TCAD} &                     & 4 & \texttt{DGTC} &                     &    &   & \\
\end{tabular}

Diese \emph{Fragmente} überführen wir direkt in einen \emph{Inzidenzgraphen} bei denen die \emph{Stellen} als Knoten und die Verbindungen in gemeinsamen \emph{Fragmenten} als Kanten übertragen werden. Um die Übersichtlichkeit halbwegs zu wahren, wurden die Kanten jeder Zusammenhangskomponente jeweils in einer Farbe markiert.

\begin{center}
	\begin{tikzpicture}[
	mycircle/.style={
		circle,
		draw=black,
		fill=gray,
		fill opacity = 0.3,
		text opacity=1,
		inner sep=0pt,
		minimum size=15pt,
		font=\tiny},
	myarrow/.style={-Stealth},
	node distance=0.6cm and 1.1cm
	]
	% erste Sequenz
	\node[mycircle] (c11) {A};
	\node[mycircle,right=of c11] (c12) {D};
	\node[mycircle,right=of c12] (c13) {G};
	\node[mycircle,right=of c13] (c14) {T};
	\node[mycircle,right=of c14] (c15) {C};
	\node[mycircle,right=of c15] (c16) {T};
	\node[mycircle,right=of c16] (c17) {C};
	\node[mycircle,right=of c17] (c18) {A};
	
	% zweite Sequenz
	\node[mycircle,below=of c11] (c21) {G};
	\node[mycircle,right=of c21] (c22) {T};
	\node[mycircle,right=of c22] (c23) {C};
	\node[mycircle,right=of c23] (c24) {A};
	\node[mycircle,right=of c24] (c25) {D};
	\node[mycircle,right=of c25] (c26) {C};
	\node[mycircle,right=of c26] (c27) {T};
	\node[mycircle,right=of c27] (c28) {C};
	\node[mycircle,right=of c28] (c29) {A};
	
	% dritte Sequenz
	\node[mycircle,below=of c21] (c31) {T};
	\node[mycircle,right=of c31] (c32) {A};
	\node[mycircle,right=of c32] (c33) {T};
	\node[mycircle,right=of c33] (c34) {C};
	\node[mycircle,right=of c34] (c35) {A};
	\node[mycircle,right=of c35] (c36) {D};
	\node[mycircle,right=of c36] (c37) {G};
	\node[mycircle,right=of c37] (c38) {G};
	
	% vierte Sequenz
	\node[mycircle,below=of c31] (c41) {D};
	\node[mycircle,right=of c41] (c42) {G};
	\node[mycircle,right=of c42] (c43) {T};
	\node[mycircle,right=of c43] (c44) {C};
	\node[mycircle,right=of c44] (c45) {A};
	\node[mycircle,right=of c45] (c46) {D};
	\node[mycircle,right=of c46] (c47) {A};
	\node[mycircle,right=of c47] (c48) {T};
	\node[mycircle,right=of c48] (c49) {C};
		
	
	% Erste Zusammenhangskomponente
	\foreach \i/\j/\txt/\p in {% start node/end node/text/position
		c16/c33//above,
		c14/c22//above,
		c33/c22//above,
		c33/c43//above,
		c22/c43//above,
		c27/c48//above,
		c14/c31//above,
		c14/c43//above,
		c16/c48//above,
		c16/c27//above}
	\draw [draw=blue, thick] (\i) -- node[sloped,font=\tiny,\p] {\txt} (\j);	
	
	% Zweite Zusammenhangskomponente
	\foreach \i/\j/\txt/\p in {% start node/end node/text/position
		c17/c34//above,
		c34/c44//above,
		c23/c34//above,
		c17/c28//above,
		c23/c44//above,
		c28/c49//above,
		c15/c32//above,
		c15/c26//above,
		c15/c44//above,
		c17/c49//above,
		c26/c47//above}
	\draw [draw=green, thick] (\i) -- node[sloped,font=\tiny,\p] {\txt} (\j);	
	
	% Dritte Zusammenhangskomponente
	\foreach \i/\j/\txt/\p in {% start node/end node/text/position
		c18/c35//above,
		c35/c24//above,
		c35/c45//above,
		c24/c45//above,
		c18/c29//above}
	\draw [draw=red, thick] (\i) -- node[sloped,font=\tiny,\p] {\txt} (\j);	
	
	% Vierte Zusammenhangskomponente
	\foreach \i/\j/\txt/\p in {% start node/end node/text/position
		c25/c36//above,
		c25/c46//above,
		c36/c46//above}
	\draw [draw=cyan, thick] (\i) -- node[sloped,font=\tiny,\p] {\txt} (\j);	
	
	% Fünfte Zusammenhangskomponente
	\foreach \i/\j/\txt/\p in {% start node/end node/text/position
		c21/c42//above,
		c13/c21//above,
		c13/c42//above}
	\draw [draw=lime, thick] (\i) -- node[sloped,font=\tiny,\p] {\txt} (\j);
	
	% Sechste Zusammenhangskomponente
	\foreach \i/\j/\txt/\p in {% start node/end node/text/position
		c12/c41//above}
	\draw [draw=magenta, thick] (\i) -- node[sloped,font=\tiny,\p] {\txt} (\j);
	\end{tikzpicture}
\end{center}

\subsection{Mehrdeutigkeiten Auflösen}

Die Zusammenhangskomponenten unseres \emph{Inzidenzgraphen} werden wir jetzt als \emph{Flussnetzwerke} interpretieren, um mit Hilfe eines Algorithmus zur Berechnung eines \emph{minimalen Schnitts} solange Kanten zu entfernen, bis alle \emph{Mehrdeutigkeiten} aufgelöst sind.

Sei $C$ eine Zusammenhangskomponente von $G_{\mathcal{F}}$, die \emph{mehrdeutig} ist, also zwei Knoten $x,y$ aus der selben Sequenz enthält. Wir wählen die \emph{mehrdeutigen} Knoten $x$ und $y$ als \emph{Quelle} und \emph{Senke} unseres \emph{Flussnetzwerks}. Die ungerichteten Kanten des \emph{Inzidenzgraphen} werden durch zwei antiparallele gerichtete Kanten ersetzt. \cite{cpm10} benutzen als Kapazitäten jeweils 1. Meiner Meinung nach wäre es hingegen sinnvoller die Kapazitäten so zu wählen, dass ähnliche Symbole seltener voneinander durch den \emph{minimalen Schnitt} und das damit einhergehende Löschen von Kanten getrennt werden. Das können wir beispielsweise erreichen, indem wir uns an den Ähnlichkeitswerten unserer Substitutionsmatrix orientieren. Weil \emph{Flussnetzwerke} nicht-negative Kapazitäten erwarten, müssen die Werte gegebenenfalls modifiziert werden, indem wir den betragsmäßig größten Eintrag der Matrix zu allen Einträgen addieren. Das bedeutet, dass bei unserer $(+3/-1)$-Substitutionsmatrix aus dem Beispiel zwischen übereinstimmenden Symbolen eine Kapazität von 6 und zwischen nicht übereinstimmendes eine von 2 benutzt wird. Ob dieser Ansatz einen Vorteil bietet werden wir später bei der Evaluierung meiner Implementierung feststellen.

Als nächstes benutzen wir einen Algorithmus zur Bestimmung des \emph{maximalen Flusses}. Wie mit dem \emph{Max-Flow-Min-Cut-Satz} gezeigt, bestimmen wir durch die Bestimmung des \emph{maximalen Flusses} auch gleichzeitig einen \emph{minimalen Schnitt}. Das ist die \enquote{schmalste} Verbindungsstelle zwischen der \emph{Quelle} und der \emph{Senke} und wir hoffen durch Löschen der dazugehörigen Kanten die \emph{Mehrdeutigkeit} aufzulösen und die dichtbesetzten Untergraphen zu erhalten. Im Original wird der Algorithmus von Edmonds-Karp benutzt, während wir stattdessen einen \emph{Push-Relabel-Algorithmus} mit Laufzeit \emph{$\oh(|V|^2\cdot \sqrt{|E|})$} für die Komplexität betrachten und in unserer Implementierung verwenden. Nachdem der \emph{minimale Schnitt} bestimmt wurde, löschen wir alle Kanten zwischen den Mengen $A$ und $B$. Auf diese Art und Weise wird unsere Zusammenhangskomponente $C$ in zwei neue Zusammenhangskomponenten $A$ und $B$ aufgeteilt. Dieses Verfahren wiederholen wir solange, bis es keine \emph{mehrdeutigen} Äquivalenzklassen mehr gibt.  \improvement{Graphik aus CPM10 mit Zusammenhangskomponenten einfügen.}


\begin{algorithm}
	\caption{Algorithmus zum Auflösen von Mehrdeutigkeiten}
	\label{alg:amb_res}
	\begin{algorithmic}[1]
		\Require \emph{Inzidenzgraph} $G_{\mathcal{F}} = (\mathcal{S},E_{\mathcal{F}})$ über einem Satz \emph{Fragmente} $\mathcal{F}$
		\Procedure{ResolveAmbiguities}{$G_{\mathcal{F}}$}
			\State{$E \gets E_{\mathcal{F}}$}
			\State{Berechne Zusammenhangskomponenten von $G_{\mathcal{F}}$}
			\While{es ex. \emph{mehrdeutige} Zusammenhangskomponente $C$ von $G_{\mathcal{F}}$}
				\While{es ex. \emph{mehrdeutige} Knoten $x,y$ aus der selben Sequenz}
					\State{Wähle Sequenz $S_i$ mit max. Anzahl an \emph{mehrdeutigen} Knoten in $C$}
					\State{Wähle $s = \text{argmin}\{\deg(v)|v \in C\: \text{und}\: v \in S_i\}$}
					\State{Wähle $t = \text{argmax}\{\deg(v)|v \in C\: \text{und}\: v \in S_i\}$}
					\State{Definiere \emph{Flussnetzwerk} auf $C$ mit \emph{Quelle} $s$ und \emph{Senke} $t$}
					\State{Benutze \textrm{PushRelabel}, um \emph{minimalen Schnitt} $C_1$ und $C_2$ zu bestimmen}
					\State{Lösche Kanten zwischen $C_1$ und $C_2$ aus $E$}
				\EndWhile
			\EndWhile
		\State{\textbf{return} \emph{nicht-mehrdeutigen} Subgraphen $(\mathcal{S},E)$ von $G_{\mathcal{F}}$}
		\EndProcedure
	\end{algorithmic}
\end{algorithm}

Unglücklicherweise können die Zusammenhangskomponenten bei \emph{Alignments} zwischen vielen langen Sequenzen sehr groß werden. \cite{cpm10} haben daher eine Grenze $k = \max\{\deg(v)| v \in \mathcal{S}\}$ eingeführt, die sukzessive gesenkt wird, bis alle \emph{Mehrdeutigkeiten} aufgelöst wurden. $k$ wird so benutzt, dass alle Kanten zwischen Knoten mit Grad $< k$ zunächst nicht betrachtet werden, sodass andere kleinere Zusammenhangskomponenten vorliegen. Als erstes werden für den reduzierten Kantensatz $E_k = \{(u,v) \in E|\min\{\deg(u),\deg(v)\}\}$ solange \emph{minimale Schnitte} berechnet und Kanten gelöscht, bis für den Graph mit weniger Kanten keine \emph{mehrdeutigen} Zusammenhangskomponenten mehr existierten. Danach wird $k$ um eins reduziert und das Vorgehen auf dem neuen Kantensatz wiederholt. Sobald $k$ auf null gesetzt wurde, hat man alle Knoten betrachtet und es liegen nur noch \emph{partielle Zuweisungsspalten} vor.

Abgesehen von der verbesserten Laufzeit scheint dieses Vorgehen aber keine Vorteile zu bieten und falls doch, werden diese in \cite{cpm10} nicht genannt. Deshalb verzichte ich auf die Grenze $k$ und gehe davon aus, dass aufgrund des deutlich effizienteren Algorithmus für den \emph{maximalen Flusses} und die Verwendung einer schnellen Graphimplementierung aus einer Bibliothek keine Laufzeitprobleme für die üblichen Anwendungsgrößen auftreten.

Bis jetzt wurde nur gesagt, dass wir \emph{mehrdeutige} Zusammenhangskomponenten auswählen und für diese jeweils einen Knoten aus der selben Sequenz als \emph{Quelle} und \emph{Senke} wählen. Abschließend müssen wir also noch festlegen in welcher Reihenfolge diese gewählt werden. Für die Zusammenhangskomponenten müssen wir keine Reihenfolge festlegen, weil diese unabhängig voneinander sind. Sei eine \emph{mehrdeutige} Zusammenhangskomponente $C$ gegeben. Dann mag es mehrere Sequenzen geben, die alle zu mehr als einem Knoten in $C$ korrespondieren. Außerdem muss es nicht unbedingt einen eindeutigen \emph{minimalen Schnitt} geben, sondern es kann mehrere solche geben. Wir entscheiden uns für dieses Vorgehen:

\begin{enumerate}[topsep=0pt,itemsep=-1ex,partopsep=1ex,parsep=1ex]
	\item Falls es mehrere Sequenzen gibt, die zwei oder mehr \emph{mehrdeutige} \emph{Stellen} in $C$ enthalten, wähle die Sequenz $S_i$ mit den meisten \emph{mehrdeutigen} Knoten in dieser Zusammenhangskomponente.
	\item Sobald $S_i$ bestimmt ist, wähle unter allen \emph{Stellen} aus dieser Sequenz in $C$ den Knoten mit dem niedrigsten Knotengrad als Quelle und den mit dem höchsten als \emph{Senke}.
	\item Sollte es mehr als einen \emph{minimalen Schnitt} geben, dann orientieren wir uns an dem Ergebnis des Algorithmus der Implementierung und wählen die Partitionierung, die sich durch die nur durch Kanten mit \emph{Restwertkapazität} 0 verbundenen Subgraphen im \emph{Restwertgraphen} ergibt. 
\end{enumerate}

\subsection{Beispiel von \textrm{ResolveAmbiguities}}

Exemplarisch betrachten wir die grüne Zusammenhangskomponente, weil diese die meisten Knoten enthält und werden an ihrem Beispiel solange den \emph{minimalen Schnitt} durchführen, bis keine \emph{Inkonsistenzen} mehr vorliegen. Die \emph{Quelle} ist grün, während die \emph{Senke} rot dargestellt ist.

\begin{center}
	\begin{tikzpicture}[
	mycircle/.style={
		circle,
		draw=black,
		fill=gray,
		fill opacity = 0.3,
		text opacity=1,
		inner sep=0pt,
		minimum size=15pt,
		font=\tiny},
	mysenke/.style={
		circle,
		draw=black,
		fill=red,
		fill opacity = 0.3,
		text opacity=1,
		inner sep=0pt,
		minimum size=15pt,
		font=\tiny},	
	myquelle/.style={
		circle,
		draw=black,
		fill=green,
		fill opacity = 0.3,
		text opacity=1,
		inner sep=0pt,
		minimum size=15pt,
		font=\tiny},
	myarrow/.style={-Stealth},
	node distance=0.6cm and 0.9cm
	]
	% erste Sequenz
	\node[mycircle] (c11) {A};
	\node[mycircle,right=of c11] (c12) {D};
	\node[mycircle,right=of c12] (c13) {G};
	\node[mycircle,right=of c13] (c14) {T};
	\node[mycircle,right=of c14] (c15) {C};
	\node[mycircle,right=of c15] (c16) {T};
	\node[mycircle,right=of c16] (c17) {C};
	\node[mycircle,right=of c17] (c18) {A};
	
	% zweite Sequenz
	\node[mycircle,below=of c11] (c21) {G};
	\node[mycircle,right=of c21] (c22) {T};
	\node[mycircle,right=of c22] (c23) {C};
	\node[mycircle,right=of c23] (c24) {A};
	\node[mycircle,right=of c24] (c25) {D};
	\node[mycircle,right=of c25] (c26) {C};
	\node[mycircle,right=of c26] (c27) {T};
	\node[mycircle,right=of c27] (c28) {C};
	\node[mycircle,right=of c28] (c29) {A};
	
	% dritte Sequenz
	\node[mycircle,below=of c21] (c31) {T};
	\node[mycircle,right=of c31] (c32) {A};
	\node[mycircle,right=of c32] (c33) {T};
	\node[mycircle,right=of c33] (c34) {C};
	\node[mycircle,right=of c34] (c35) {A};
	\node[mycircle,right=of c35] (c36) {D};
	\node[mycircle,right=of c36] (c37) {G};
	\node[mycircle,right=of c37] (c38) {G};
	
	% vierte Sequenz
	\node[mycircle,below=of c31] (c41) {D};
	\node[mycircle,right=of c41] (c42) {G};
	\node[mycircle,right=of c42] (c43) {T};
	\node[mysenke,right=of c43] (c44) {C};
	\node[mycircle,right=of c44] (c45) {A};
	\node[mycircle,right=of c45] (c46) {D};
	\node[myquelle,right=of c46] (c47) {A};
	\node[mycircle,right=of c47] (c48) {T};
	\node[mycircle,right=of c48] (c49) {C};

	% Zweite Zusammenhangskomponente
	\foreach \i/\j/\txt/\p in {% start node/end node/text/position
		c17/c34/{0,6}/above,
		c15/c32/{0,2}/above,
		c34/c44/{0,6}/above,
		c23/c34/{0,6}/above,
		c17/c28/{0,6}/above,
		c23/c44/{0,6}/above,
		c28/c49/{0,6}/above,
		c15/c26/{2/6}/above,
		c15/c44/{2,6}/above,
		c17/c49/{0,6}/below}
	\draw [Stealth-Stealth, draw=black, double=white, thick] (\i) -- node[sloped,font=\tiny,\p] {\txt} (\j);	
	
	\foreach \i/\j/\txt/\p in {% start node/end node/text/position
		c26/c47/{2/2}/above}
	\draw [Stealth-Stealth, draw=black, double=white, dotted, thick] (\i) -- node[sloped,font=\tiny,\p] {\txt} (\j);	
	\end{tikzpicture}
\end{center}

Sowohl $S_2$, als auch $S_4$ haben drei Knoten in $C$. Wir wählen hier $S_4[7]$ als \emph{Quelle} und $S_4[4]$ als \emph{Senke}, weil diese den kleinsten bzw. größten Knotengrad haben. In diesem Fall passiert nichts spannendes, weil der \emph{minimale Schnitt} aufgrund der Kante mit Kapazität 2 direkt die \emph{Quelle} vom Rest der Zusammenhangskomponente trennt.

Im nächsten Schritt hat $S_2$ mit 3 die meisten Knoten in $C$. Da alle zwei der Knoten Grad 2 haben, entscheiden wir uns für $S_2[6]$ als \emph{Quelle} und $S_2[8]$ als \emph{Senke}. Auch hier wird direkt die erste Kante an der \emph{Quelle} gelöscht.

\begin{center}
	\begin{tikzpicture}[
	mycircle/.style={
		circle,
		draw=black,
		fill=gray,
		fill opacity = 0.3,
		text opacity=1,
		inner sep=0pt,
		minimum size=15pt,
		font=\tiny},
	mysenke/.style={
		circle,
		draw=black,
		fill=red,
		fill opacity = 0.3,
		text opacity=1,
		inner sep=0pt,
		minimum size=15pt,
		font=\tiny},	
	myquelle/.style={
		circle,
		draw=black,
		fill=green,
		fill opacity = 0.3,
		text opacity=1,
		inner sep=0pt,
		minimum size=15pt,
		font=\tiny},
	myarrow/.style={-Stealth},
	node distance=0.6cm and 0.9cm
	]
	% erste Sequenz
	\node[mycircle] (c11) {A};
	\node[mycircle,right=of c11] (c12) {D};
	\node[mycircle,right=of c12] (c13) {G};
	\node[mycircle,right=of c13] (c14) {T};
	\node[mycircle,right=of c14] (c15) {C};
	\node[mycircle,right=of c15] (c16) {T};
	\node[mycircle,right=of c16] (c17) {C};
	\node[mycircle,right=of c17] (c18) {A};
	
	% zweite Sequenz
	\node[mycircle,below=of c11] (c21) {G};
	\node[mycircle,right=of c21] (c22) {T};
	\node[mysenke,right=of c22] (c23) {C};
	\node[mycircle,right=of c23] (c24) {A};
	\node[mycircle,right=of c24] (c25) {D};
	\node[myquelle,right=of c25] (c26) {C};
	\node[mycircle,right=of c26] (c27) {T};
	\node[mycircle,right=of c27] (c28) {C};
	\node[mycircle,right=of c28] (c29) {A};
	
	% dritte Sequenz
	\node[mycircle,below=of c21] (c31) {T};
	\node[mycircle,right=of c31] (c32) {A};
	\node[mycircle,right=of c32] (c33) {T};
	\node[mycircle,right=of c33] (c34) {C};
	\node[mycircle,right=of c34] (c35) {A};
	\node[mycircle,right=of c35] (c36) {D};
	\node[mycircle,right=of c36] (c37) {G};
	\node[mycircle,right=of c37] (c38) {G};
	
	% vierte Sequenz
	\node[mycircle,below=of c31] (c41) {D};
	\node[mycircle,right=of c41] (c42) {G};
	\node[mycircle,right=of c42] (c43) {T};
	\node[mycircle,right=of c43] (c44) {C};
	\node[mycircle,right=of c44] (c45) {A};
	\node[mycircle,right=of c45] (c46) {D};
	\node[mycircle,right=of c46] (c47) {A};
	\node[mycircle,right=of c47] (c48) {T};
	\node[mycircle,right=of c48] (c49) {C};
	
	% Zweite Zusammenhangskomponente
	\foreach \i/\j/\txt/\p in {% start node/end node/text/position
		c17/c34/{0,6}/above,
		c15/c32/{0,2}/above,
		c34/c44/{0,6}/above,
		c23/c34/{0,6}/above,
		c17/c28/{0,6}/above,
		c23/c44/{6,6}/above,
		c28/c49/{0,6}/above,
		c15/c44/{6,6}/above,
		c17/c49/{0,6}/below}
	\draw [Stealth-Stealth, draw=black, double=white, thick] (\i) -- node[sloped,font=\tiny,\p] {\txt} (\j);	
	
	\foreach \i/\j/\txt/\p in {% start node/end node/text/position
		c26/c15/{6/6}/above}
	\draw [Stealth-Stealth, draw=black, double=white, dotted, thick] (\i) -- node[sloped,font=\tiny,\p] {\txt} (\j);	
	\end{tikzpicture}
\end{center}

Gehen nun wieder zu Sequenz $S_4$ und wählen $S_4[9]$ als \emph{Quelle} und $S_4[4]$ als \emph{Senke}. Hier haben wir zwei relativ dichte Subgraphen, die nur durch die eine Kante $(S_1[7] \rightarrow S_3[4])$ miteinander verbunden sind. Diese löschen wir und danach gibt es in der rechten Zusammenhangskomponente keine \emph{Mehrdeutigkeiten} mehr. 

\begin{center}
	\begin{tikzpicture}[
	mycircle/.style={
		circle,
		draw=black,
		fill=gray,
		fill opacity = 0.3,
		text opacity=1,
		inner sep=0pt,
		minimum size=15pt,
		font=\tiny},
	mysenke/.style={
		circle,
		draw=black,
		fill=red,
		fill opacity = 0.3,
		text opacity=1,
		inner sep=0pt,
		minimum size=15pt,
		font=\tiny},	
	myquelle/.style={
		circle,
		draw=black,
		fill=green,
		fill opacity = 0.3,
		text opacity=1,
		inner sep=0pt,
		minimum size=15pt,
		font=\tiny},
	myarrow/.style={-Stealth},
	node distance=0.6cm and 0.9cm
	]
	% erste Sequenz
	\node[mycircle] (c11) {A};
	\node[mycircle,right=of c11] (c12) {D};
	\node[mycircle,right=of c12] (c13) {G};
	\node[mycircle,right=of c13] (c14) {T};
	\node[mycircle,right=of c14] (c15) {C};
	\node[mycircle,right=of c15] (c16) {T};
	\node[mycircle,right=of c16] (c17) {C};
	\node[mycircle,right=of c17] (c18) {A};
	
	% zweite Sequenz
	\node[mycircle,below=of c11] (c21) {G};
	\node[mycircle,right=of c21] (c22) {T};
	\node[mycircle,right=of c22] (c23) {C};
	\node[mycircle,right=of c23] (c24) {A};
	\node[mycircle,right=of c24] (c25) {D};
	\node[mycircle,right=of c25] (c26) {C};
	\node[mycircle,right=of c26] (c27) {T};
	\node[mycircle,right=of c27] (c28) {C};
	\node[mycircle,right=of c28] (c29) {A};
	
	% dritte Sequenz
	\node[mycircle,below=of c21] (c31) {T};
	\node[mycircle,right=of c31] (c32) {A};
	\node[mycircle,right=of c32] (c33) {T};
	\node[mycircle,right=of c33] (c34) {C};
	\node[mycircle,right=of c34] (c35) {A};
	\node[mycircle,right=of c35] (c36) {D};
	\node[mycircle,right=of c36] (c37) {G};
	\node[mycircle,right=of c37] (c38) {G};
	
	% vierte Sequenz
	\node[mycircle,below=of c31] (c41) {D};
	\node[mycircle,right=of c41] (c42) {G};
	\node[mycircle,right=of c42] (c43) {T};
	\node[mysenke,right=of c43] (c44) {C};
	\node[mycircle,right=of c44] (c45) {A};
	\node[mycircle,right=of c45] (c46) {D};
	\node[mycircle,right=of c46] (c47) {A};
	\node[mycircle,right=of c47] (c48) {T};
	\node[myquelle,right=of c48] (c49) {C};
	
	% Zweite Zusammenhangskomponente
	\foreach \i/\j/\txt/\p in {% start node/end node/text/position
		c15/c32/{0,2}/above,
		c34/c44/{0,6}/above,
		c23/c34/{6,6}/above,
		c17/c28/{0,6}/above,
		c23/c44/{6,6}/above,
		c28/c49/{0,6}/above,
		c15/c44/{6,6}/above,
		c17/c49/{6,6}/below}
	\draw [Stealth-Stealth, draw=black, double=white, thick] (\i) -- node[sloped,font=\tiny,\p] {\txt} (\j);	
	
	\foreach \i/\j/\txt/\p in {% start node/end node/text/position
		c17/c34/{6/6}/above}
	\draw [Stealth-Stealth, draw=black, double=white, dotted, thick] (\i) -- node[sloped,font=\tiny,\p] {\txt} (\j);	
	\end{tikzpicture}
\end{center}

Abschließend wird noch die Verbindung zwischen dem \texttt{A} aus der zweiten Sequenz und dem \texttt{C} an der fünften Stelle der ersten gelöscht, was hier nicht mehr graphisch dargestellt wurde. Danach sind alle \emph{Mehrdeutigkeiten} aufgelöst und es liegen nur noch \emph{partielle Zuweisungsspalten} vor.

Wenn wir den Algorithmus auch auf allen anderen Zusammenhangskomponenten anwenden, dann kommen wir zu folgendem \emph{Inzidenzgraphen} ohne \emph{Mehrdeutigkeiten}:
 
\begin{center}
	\begin{tikzpicture}[
	mycircle/.style={
		circle,
		draw=black,
		fill=gray,
		fill opacity = 0.3,
		text opacity=1,
		inner sep=0pt,
		minimum size=15pt,
		font=\tiny},
	myarrow/.style={-Stealth},
	node distance=0.6cm and 1.1cm
	]
	% erste Sequenz
	\node[mycircle] (c11) {A};
	\node[mycircle,right=of c11] (c12) {D};
	\node[mycircle,right=of c12] (c13) {G};
	\node[mycircle,right=of c13] (c14) {T};
	\node[mycircle,right=of c14] (c15) {C};
	\node[mycircle,right=of c15] (c16) {T};
	\node[mycircle,right=of c16] (c17) {C};
	\node[mycircle,right=of c17] (c18) {A};
	
	% zweite Sequenz
	\node[mycircle,below=of c11] (c21) {G};
	\node[mycircle,right=of c21] (c22) {T};
	\node[mycircle,right=of c22] (c23) {C};
	\node[mycircle,right=of c23] (c24) {A};
	\node[mycircle,right=of c24] (c25) {D};
	\node[mycircle,right=of c25] (c26) {C};
	\node[mycircle,right=of c26] (c27) {T};
	\node[mycircle,right=of c27] (c28) {C};
	\node[mycircle,right=of c28] (c29) {A};
	
	% dritte Sequenz
	\node[mycircle,below=of c21] (c31) {T};
	\node[mycircle,right=of c31] (c32) {A};
	\node[mycircle,right=of c32] (c33) {T};
	\node[mycircle,right=of c33] (c34) {C};
	\node[mycircle,right=of c34] (c35) {A};
	\node[mycircle,right=of c35] (c36) {D};
	\node[mycircle,right=of c36] (c37) {G};
	\node[mycircle,right=of c37] (c38) {G};
	
	% vierte Sequenz
	\node[mycircle,below=of c31] (c41) {D};
	\node[mycircle,right=of c41] (c42) {G};
	\node[mycircle,right=of c42] (c43) {T};
	\node[mycircle,right=of c43] (c44) {C};
	\node[mycircle,right=of c44] (c45) {A};
	\node[mycircle,right=of c45] (c46) {D};
	\node[mycircle,right=of c46] (c47) {A};
	\node[mycircle,right=of c47] (c48) {T};
	\node[mycircle,right=of c48] (c49) {C};
	
	
	% Erste Zusammenhangskomponente
	\foreach \i/\j/\txt/\p in {% start node/end node/text/position
		c14/c22//above,
		c33/c22//above,
		c33/c43//above,
		c22/c43//above,
		c27/c48//above,
		c14/c43//above,
		c16/c48//above,
		c16/c27//above}
	\draw [draw=blue, thick] (\i) -- node[sloped,font=\tiny,\p] {\txt} (\j);	
	
	% Zweite Zusammenhangskomponente
	\foreach \i/\j/\txt/\p in {% start node/end node/text/position
		c34/c44//above,
		c23/c34//above,
		c17/c28//above,
		c23/c44//above,
		c28/c49//above,
		c15/c44//above,
		c17/c49//above}
	\draw [draw=green, thick] (\i) -- node[sloped,font=\tiny,\p] {\txt} (\j);	
	
	% Dritte Zusammenhangskomponente
	\foreach \i/\j/\txt/\p in {% start node/end node/text/position
		c18/c35//above,
		c35/c24//above,
		c35/c45//above,
		c24/c45//above}
	\draw [draw=red, thick] (\i) -- node[sloped,font=\tiny,\p] {\txt} (\j);	
	
	% Vierte Zusammenhangskomponente
	\foreach \i/\j/\txt/\p in {% start node/end node/text/position
		c25/c36//above,
		c25/c46//above,
		c36/c46//above}
	\draw [draw=cyan, thick] (\i) -- node[sloped,font=\tiny,\p] {\txt} (\j);	
	
	% Fünfte Zusammenhangskomponente
	\foreach \i/\j/\txt/\p in {% start node/end node/text/position
		c21/c42//above,
		c13/c21//above,
		c13/c42//above}
	\draw [draw=lime, thick] (\i) -- node[sloped,font=\tiny,\p] {\txt} (\j);
	
	% Sechste Zusammenhangskomponente
	\foreach \i/\j/\txt/\p in {% start node/end node/text/position
		c12/c41//above}
	\draw [draw=magenta, thick] (\i) -- node[sloped,font=\tiny,\p] {\txt} (\j);
	\end{tikzpicture}
\end{center}

Obwohl in diesem Graphen keine \emph{Mehrdeutigkeiten} mehr existieren, ist die Zuordnung noch nicht \emph{konsistent}. Das liegt an Überkreuzungen, wie der zwischen $(S_1[8] \rightarrow S_3[5] \rightarrow S_4[5])$ (rot) und $(S_1[7] \rightarrow S_4[9])$ (grün). 

\subsection{Komplexität}

Die Laufzeit von \textrm{ResolveAmbiguities} wird von der Laufzeit zur Berechnung des \emph{minimalen Schnitts} durch \textrm{PushRelabel} dominiert \citep{cpm10}. Diese hängt von der Größe unserer Zusammenhangskomponenten ab. Im schlimmsten Fall besteht der \emph{Inzidenzgraph} aus einer einzigen Zusammenhangskomponente bei der jeder Knoten mit mindestens einem Knoten aus jeder anderen Sequenz verbunden ist. In diesem Fall muss sie in vielen Schritten zerkleinert werden muss, bis nur noch \emph{partielle Zuweisungsspalten} übrig bleiben. Eine Zusammenhangskomponente kann $n\cdot L$ Knoten und $\oh(n^2\cdot L)$ Kanten haben. Das liegt daran, dass wir als Grundlage unsere paarweisen \emph{Alignments} benutzen, bei denen jede \emph{Stelle} für jede andere Sequenz mit höchstens einer \emph{Stelle} aus dieser verbunden ist. 

\textrm{PushRelabel} hat eine Komplexität von $\oh(|V|^2\cdot \sqrt{|E|})$, um einen \emph{minimalen Schnitt} zu berechnen. Weil $|V| \in \oh(n\cdot L)$ und $|E| \in \oh(n^2\cdot L)$ gelten, braucht man für einen Durchlauf also $\oh(n^3\cdot L^{5/2})$. Wenn wir Pech haben, entfernt jeder Aufruf unseres Algorithmus für den \emph{maximalen Fluss} nur einen einzigen Knoten aus der Zusammenhangskomponente. Das resultiert in $n\cdot L$ Aufrufen und einer Gesamtlaufzeit von $\oh(n^4\cdot L^{7/2})$.

Möglicherweise lässt sich eine bessere Laufzeit erreichen, wenn alle Kanten eine uniforme Kapazität haben, wie das bei der Variante der Autoren der Fall ist. Es lässt sich zeigen, dass das ein deutlich leichteres Problem ist \citep{gt14} und nach Karzanov und Even hat eine Variante des Algorithmus von Dinic in diesem Fall gute Laufzeiten. 

Glücklicherweise sind die Zusammenhangskomponenten im Allgemeinen nicht so groß und in den meisten Fällen trennt man mit dem \emph{minimalen Schnitt} auch nicht nur einzelne Knoten ab. Bei Testläufen auf der Protein-Referenzdatenbank BAliBASE haben \cite{cpm10} die Referenzmenge RV12 genauer betrachtet. RV12 besteht aus 88 Sequenzfamilien, die im Schnitt zehn Sequenzen enthielt. Messungen haben ergeben, dass die \emph{Inzidenzgraphen} auf diesen Sequenzfamilien im Schnitt 2877 Knoten und 10952 in 223 Zusammenhangskomponenten enthalten haben. Auf diesen Sequenzen lassen sich in guten Laufzeiten von weniger als einer Minute multiple \emph{Sequenzalignments} berechnen. Anders sah es auf der Sequenzfamilie BB30003 aus. Diese besteht aus 142 Sequenzen und resultiert in einem monströsen \emph{Inzidenzgraphen}, der nur aus einer einzigen Zusammenhangskomponente besteht. Nach einer Laufzeit von 20 Stunden ohne Ergebnis wurde der Lauf auf Graphen, der $1,5\cdot 10^6$ Kanten enthält, erfolglos abgebrochen. Erst mit einer Begrenzung des minimalen \emph{Fragmentgewichts} auf 4 ließ sich in 13 Stunden ein Ergebnis erzielen. 

\section{Sukzessionsgraphen und der Algorithmus von Pitschi}

\subsection{Aufbau des Sukzessionsgraphen}

Wie wir gesehen haben, ist die Menge an Zuweisungen nach dem \textrm{ResolveAmbiguities}-Aufruf noch nicht \emph{konsistent}, aber es liegen keine \emph{Mehrdeutigkeiten} mehr vor. Das zwingt uns dazu weitere Verbindungen aus der Äquivalenzrelation zu löschen, bis diese ein \emph{Alignment}, also \emph{konsistent} ist. Wir führen dazu eine Datenstruktur ein, die die Zusammenhangskomponenten des reduzierten \emph{Inzidenzgraphen} nach ihrer Ordnung in den beteiligten Sequenzen ordnet.

\begin{definition}[Sukzessionsgraph]
	Es sei eine Menge an Sequenzen $S$ mit \emph{Stellenraum} $\mathcal{S}$ gegeben. Für diese Sequenzen liegt eine Menge $\mathcal{C}$ von Zusammenhangskomponenten vor, die alle \emph{partielle Zuweisungsspalten} sind. Dann definieren wir den \emph{Sukzessionsgraph} $SG(\mathcal{C}) = (\mathcal{C},E)$ als gerichteten, gewichteten Graphen. In $SG(\mathcal{C})$ fügen wir genau dann eine Kante $e = (C,C')$ für $C,C' \in \mathcal{C}$ ein, wenn eine Sequenz $S_i \in S$ existiert, für die \emph{Stellen} $s = (i,p) \in C$ und $s' = (i,p') \in C'$ vorliegen mit $p < p'$ und außerdem keine \emph{Stelle} $s = (i,p'')$ in einem anderen Knoten $C''$ mit $p < p'' < p'$ vorhanden ist. Das Gewicht von $e$ ist die Anzahl an Sequenzen für die die obige Bedingung gilt. Des Weiteren setzen wir voraus, dass alle Zusammenhangskomponenten mindestens zwei Knoten enthalten.
	Außerdem fügen wir zwei zusätzliche Knoten $v_{start}$ und $v_{end}$ ein. $v_{start}$ ist mit allen Knoten verbunden, die die erste \emph{Stelle} einer Sequenz enthalten. Zusätzlich sind alle Knoten, die die letzte \emph{Stelle} einer Sequenz enthalten mit $v_{end}$ über eine Kante verbunden.
\end{definition}


Wie man sieht sind zwei Knoten $C$ und $C'$ aus dem \emph{Sukzessionsgraphen} genau dann miteinander verbunden, wenn es in diesen \emph{Stellen} aus der selben Sequenz gibt, bei denen die aus $C$ links von der in $C'$ stehen. Die zusätzlich eingefügten Knoten $v_{start}$ und $v_{end}$ brauchen wir im Algorithmus von Pitschi, um die Konnektivität des Graphen zu erhalten, wenn wir Kanten aus dem Graph löschen. Das werden wir später tun, um \emph{Inkonsistenzen} zu entfernen. Setzen wir nicht voraus, dass die Knoten von $SG$ mindestens zwei \emph{Stellen} enthalten, dann liefert der Algorithmus von Pitschi suboptimale Ergebnisse, wie wir später sehen werden. \cite{cpm10} und \cite{pdc10} sind an dieser Stelle nicht eindeutig. 

Als Beispiel konstruieren wir aus dem reduzierten \emph{Inzidenzgraphen} des letzten Schritts einen \emph{Sukzessionsgraphen}. Die Knotenfarben wurden in der Farbe der Kanten des alten Graphen gewählt.

\begin{center}
	\begin{tikzpicture}[
	mycircle/.style={
		circle,
		draw=black,
		fill opacity = 0.3,
		text opacity=1,
		inner sep=0pt,
		minimum size=15pt,
		font=\tiny},
	myarrow/.style={-Stealth},
	node distance=0.6cm and 1.1cm
	]
	
	\node[mycircle,fill=gray] at (0.5,-0.5) (c1) {$v_{start}$};
	\node[mycircle,fill=magenta] at (2,1.5) (c2) {\begin{tabular}{c}
		$(1,2)$ \\ $(4,1)$
		\end{tabular}};
	\node[mycircle,fill=lime] at (4,1.5) (c3) {\begin{tabular}{c}
		$(1,3)$ \\ $(2,1)$ \\ $(4,2)$
		\end{tabular}};
	\node[mycircle,fill=blue] at (4,-0.5) (c4) {\begin{tabular}{c}
		$(1,4)$ \\ $(2,2)$ \\ $(3,3)$ \\ $(4,3)$
		\end{tabular}};
	\node[mycircle,fill=green] at (6,-0.5) (c5) {\begin{tabular}{c}
		$(1,5)$ \\ $(2,3)$ \\ $(3,4)$ \\ $(4,4)$
		\end{tabular}};
	\node[mycircle,fill=red] at (8,-0.5) (c6) {\begin{tabular}{c}
		$(1,8)$ \\ $(2,4)$ \\ $(3,5)$ \\ $(4,5)$
		\end{tabular}};
	\node[mycircle,fill=cyan] at (10,-0.5) (c7) {\begin{tabular}{c}
		$(2,5)$ \\ $(3,6)$ \\ $(4,6)$
		\end{tabular}};
	\node[mycircle,fill=blue] at (8,1.5) (c8) {\begin{tabular}{c}
		$(1,6)$ \\ $(2,7)$ \\ $(4,8)$
		\end{tabular}};
	\node[mycircle,fill=green] at (10,1.5) (c9) {\begin{tabular}{c}
		$(1,7)$ \\ $(2,8)$ \\ $(4,9)$
		\end{tabular}};
	\node[mycircle, fill=gray] at (12,-0.5) (c10) {$v_{end}$};
	% Kanten ohne Cycle
	\foreach \i/\j/\txt/\p in {% start node/end node/text/position
		c1/c2/2/above,
		c1/c3/1/below,
		c2/c3/2/above,	
		c1/c4/1/above,
		c3/c4/3/above,
		c4/c5/4/above,
		c5/c6/3/above,
		c5/c8/1/above,
		c9/c10/2/above,
		c7/c10/1/above}	
	\draw [myarrow] (\i) -- node[sloped,font=\tiny,\p] {\txt} (\j);
	
	\draw [myarrow] (c6) to[bend right] node[sloped,font=\tiny,below] {1} (c10);
		
	% Cycle%	
	\foreach \i/\j/\txt/\p in {% start node/end node/text/position
		c6/c7/3/above,
		c8/c9/3/above}
	\draw [myarrow, draw=red] (\i) -- node[sloped,font=\tiny,\p] {\txt} (\j);	

	\draw [myarrow, draw=red] (c7) -- node[sloped,font=\tiny,above, pos=0.25] {2} (c8);	
	\draw [myarrow, draw=red] (c9) -- node[sloped,font=\tiny,above, pos=0.25] {1} (c6);		
	\end{tikzpicture}
\end{center}

\begin{lemma}
	Die Menge $\mathcal{C}$ ist genau dann \emph{konsistent}, wenn $SG(\mathcal{C})$ ein gerichteter, azyklischer Graph (DAG) ist.
\end{lemma}

\begin{beweis}
	\bewhin \hspace{2pt} Sei $\mathcal{C}$ eine \emph{konsistente} Menge von Zusammenhangskomponenten mit der induzierten Äquivalenzrelation $\mathcal{R}$. Dann folgt aus $x \preceq_{\mathcal{R}} y$ auch $x \preceq y$ für alle Sequenzen. Angenommen es gibt in $SG(\mathcal{C})$ einen Zyklus. Dann gibt es Knoten $C, C'$ mit \emph{Stellen} $s_1, s_2 \in C$ und $s_1',s_2' \in C'$ mit folgenden Eigenschaften:
	\begin{enumerate}[topsep=0pt,itemsep=-1ex,partopsep=1ex,parsep=1ex]
		\item $s_1, s_1' \in S_1$ und $s_2,s_2' \in S_2$ 
		\item $pos(s_1) < pos(s_1')$ und $pos(s_2') < pos(s_2)$
	\end{enumerate}
	Diese muss es geben, sonst gäbe es keinen Zyklus im Graphen. Für diese \emph{Stellen} gelten $s_1' \mathcal{R} s_2'$, $s_2' \preceq s_2$ und $s_2 \mathcal{R} s_1$, woraus $s_1' \preceq_{\mathcal{R}} s_1$ folgt. Es gilt aufgrund von $pos(s_1) < pos(s_1')$, weshalb $s_1 \npreceq s_1'$ folgt. Das steht im Widerspruch dazu, dass $\mathcal{C}$ \emph{konsistent} war.
	
	\bewrueck \hspace{2pt} Wir benutzen einen Kontrapositionsbeweis und es sei eine \emph{inkonsistente} Menge $\mathcal{C}$ gegeben. Dann existiert eine Sequenz $S_1$, die \emph{Stellen} $s_1,s_1' \in S_1$ mit folgenden Eigenschaften enthält:
	\begin{enumerate}[topsep=0pt,itemsep=-1ex,partopsep=1ex,parsep=1ex]
		\item $s_1 \preceq_{\mathcal{R}} s_1'$
		\item $s_1 \npreceq s_1'$
	\end{enumerate}
	Aus $s_1 \preceq_{\mathcal{R}} s_1'$ folgt, dass es zwei Knoten $C$ und $C'$ gibt mit $s_1 \in C$ und $s_1' \in C'$, die in $SG(\mathcal{C})$ über einen Pfad verbunden sind. Der Pfad kann aber nicht über eine Kante laufen, die zu $S_1$ gehört, denn $s_1 \npreceq s_1'$ gilt. Aus dieser Bedingung folgt aber, dass einen Pfad von $C'$ zu $C'$ geben muss. Da es sowohl von $C$ nach $C'$, als auch von $C'$ nach $C$ Pfade gibt, kann $SG(\mathcal{C})$ nicht azyklisch sein.
\end{beweis}

\subsection{Der Algorithmus von Pitschi}

Aufgrund des gerade gezeigten Lemmas folgt, dass wir eine \emph{konsistente} Menge von \emph{partiellen Zuweisungsspalten} finden, wenn deren \emph{Sukzessionsgraph} azyklisch ist. Der Algorithmus von \cite{pdc10} sieht zwei Schritte vor, um aus dem potentiell zyklischen \emph{Sukzessionsgraphen} einen kreisfreien zu konstruieren, der mit einer \emph{konsistenten} Menge von \emph{partiellen Zuweisungsspalten} korrespondiert:

\begin{enumerate}[topsep=0pt,itemsep=-1ex,partopsep=1ex,parsep=1ex]
	\item Lösche Kanten aus dem \emph{Sukzessionsgraphen}, bis dieser kreisfrei ist.
	\item Benutze den so entstandenen DAG, um zu entscheiden, welche \emph{Stellen} aus den jeweiligen Knoten gelöscht werden müssen, um \emph{Inkonsistenzen} zu entfernen.
\end{enumerate}

\subsubsection{Entfernen der Kanten}

Wir beginnen mit der Transformation des zyklischen Graphen in einen azyklischen, indem wir Kanten entfernen. Optimal wäre es Kanten mit einer minimalen Summe von Kantengewichten zu entfernen. Leider ist dieses Problem, das auch als \enquote{minimal weighted feedback arc set}-Problem bekannt ist, NP-schwer. Als Folge bedienen wir uns einfach einer einfachen Heuristik, indem wir sukzessive eine Grenze $k$ erhöhen und in jedem Schritt alle Kanten mit einem Gewicht kleiner als $k$ löschen, bis es keinen Zyklus im Graphen mehr gibt. Formal definieren wir die Menge an Kanten, die mindestens das Gewicht $k\in \mathbb{N}$ hat als 
\begin{equation}
\begin{split}
	& E_k \coloneqq \{(u,v)\in E | w(u,v) > k\: \text{oder}\: u = v_{start}\: \text{oder}\: v = v_{end}\}\: \text{mit} \\
	& k^* \coloneqq \min\{k > 0 | (V,E_k)\: \text{ist azyklisch}\}
\end{split}
\end{equation}

Für den Fall, dass durch das Löschen von Kanten der Graph nicht mehr zusammenhängend ist, fügen wir für jede Zusammenhangskomponente und jede an dieser beteiligten Sequenz eine Kante vom Startzustand $v_{start}$ zum Knoten mit der kleinsten \emph{Stelle} aus der gewählten Sequenz ein. Analog gehen wir mit den größten \emph{Stellen} und dem Endzustand $v_{end}$ vor. Diese Menge dieser Kanten nennen wir $E_c$. Auf diese Weise ist sichergestellt, dass der azyklische Graph $G^* = (V,E_{k^*}\cup E_c)$ zusammenhängend ist und über jeden Knoten ein Pfad vom Start- zum Endzustand führt.

\subsubsection{Beispiel zum Entfernen von Kanten}

Bei unserem Beispiel ist glücklicherweise nur Zyklus vorhanden, der bereits in $(V,E_1)$ nicht mehr vorhanden ist. Wir löschen also alle Kanten mit Kantengewicht 1 oder weniger, die nicht zum Start- oder Endknoten gehören. Entfernte Kanten wurden gestrichelt dargestellt. Das sieht dann so aus:  

\begin{center}
	\begin{tikzpicture}[
	mycircle/.style={
		circle,
		draw=black,
		fill opacity = 0.3,
		text opacity=1,
		inner sep=0pt,
		minimum size=15pt,
		font=\tiny},
	myarrow/.style={-Stealth},
	node distance=0.6cm and 1.1cm
	]
	
	\node[mycircle,fill=gray] at (0.5,-0.5) (c1) {$v_{start}$};
	\node[mycircle,fill=magenta] at (2,1.5) (c2) {\begin{tabular}{c}
		$(1,2)$ \\ $(4,1)$
		\end{tabular}};
	\node[mycircle,fill=lime] at (4,1.5) (c3) {\begin{tabular}{c}
		$(1,3)$ \\ $(2,1)$ \\ $(4,2)$
		\end{tabular}};
	\node[mycircle,fill=blue] at (4,-0.5) (c4) {\begin{tabular}{c}
		$(1,4)$ \\ $(2,2)$ \\ $(3,3)$ \\ $(4,3)$
		\end{tabular}};
	\node[mycircle,fill=green] at (6,-0.5) (c5) {\begin{tabular}{c}
		$(1,5)$ \\ $(2,3)$ \\ $(3,4)$ \\ $(4,4)$
		\end{tabular}};
	\node[mycircle,fill=red] at (8,-0.5) (c6) {\begin{tabular}{c}
		$(1,8)$ \\ $(2,4)$ \\ $(3,5)$ \\ $(4,5)$
		\end{tabular}};
	\node[mycircle,fill=cyan] at (10,-0.5) (c7) {\begin{tabular}{c}
		$(2,5)$ \\ $(3,6)$ \\ $(4,6)$
		\end{tabular}};
	\node[mycircle,fill=blue] at (8,1.5) (c8) {\begin{tabular}{c}
		$(1,6)$ \\ $(2,7)$ \\ $(4,8)$
		\end{tabular}};
	\node[mycircle,fill=green] at (10,1.5) (c9) {\begin{tabular}{c}
		$(1,7)$ \\ $(2,8)$ \\ $(4,9)$
		\end{tabular}};
	\node[mycircle, fill=gray] at (12,-0.5) (c10) {$v_{end}$};
	% Kanten ohne Cycle
	\foreach \i/\j/\txt/\p in {% start node/end node/text/position
		c1/c2/2/above,
		c1/c3/1/below,
		c2/c3/2/above,	
		c1/c4/1/above,
		c3/c4/3/right,
		c4/c5/4/above,
		c5/c6/3/above,
		c9/c10/2/above,
		c7/c10/1/above}	
	\draw [myarrow] (\i) -- node[font=\tiny,\p] {\txt} (\j);
	
	\draw [myarrow] (c6) to[bend right] node[font=\tiny,below] {1} (c10);

	\draw [myarrow,dotted] (c5) -- node[font=\tiny,above] {1} (c8);
	
	% Cycle%	
	\foreach \i/\j/\txt/\p in {% start node/end node/text/position
		c6/c7/3/above,
		c8/c9/3/above}
	\draw [myarrow, draw=red] (\i) -- node[font=\tiny,\p] {\txt} (\j);	
	
	\draw [myarrow, draw=red] (c7) -- node[font=\tiny,above, pos=0.25] {2} (c8);	
	\draw [myarrow, draw=red, dotted] (c9) -- node[font=\tiny,above, pos=0.25] {1} (c6);		
	\end{tikzpicture}
\end{center}

\subsubsection{Entfernen von Stellen}

Als nächstes lernen wir einen von \cite{pdc10} entwickelten Algorithmus kennen, der eine minimale Anzahl an \emph{Stellen} aus dem verkleinerten \emph{Sukzessionsgraphen} $G^{*}$ löscht, um alle \emph{Inkonsistenzen} zu entfernen. Dabei orientiert sich der Algorithmus an der linearen Halbordnung $\preceq$ über $\mathcal{S}$ und der Halbordnung auf dem DAG, die ersterer angepasst werden soll. Wir bezeichnen die Halbordnung auf $\mathcal{C}$, die durch den Graph $G^{*}$ induziert wird, als $\preceq^{*}$.

Sei eine Sequenz $S_i \in S$ gegeben. Dann ist $\mathcal{C}_{S_i}$ die Teilmenge von Zusammenhangskomponenten aus $\mathcal{C}$, die \emph{Stellen} aus $S_i$ enthalten. Wir definieren außerdem eine Beschränkung der Knoten aus $G^{*}$ von $\mathcal{C}_{S_i}$ als $V_{S_i} = \mathcal{C}_{S_i} \cup \{v_{start},v_{end}\}$. Auf diesen existiert eine Ordnung $\preceq_{S_i}$, die durch die natürliche Ordnung auf $S_i$ gegeben ist, und die Halbordnung $\preceq_{S_i}^{*}$ des Graphen mit reduzierter Knotenmenge. Wir definieren $\mathcal{R}_{S_i} = \preceq_{S_i}\, \cap\, \preceq_{S_i}^{*}$. Diese Relation entspricht genau den Verbindungen der transitiven Hülle im ursprünglichen Graph, wenn man $G^{*}$ auf die Knoten aus $\mathcal{C}_{S_i}$ beschränkt. Sei dafür $G^{+} = (V,E^{+})$ die transitive Hülle $TC(G^{*})$ des DAG. Weil $G^{*}$ keine Zyklen enthält, kann auch $TC(G^{*})$ keine enthalten. Wir können $G^{*}$ auf jede unserer Sequenzen $S_i$ beschränken, indem wir den reduzierten Knotensatz $V_{S_i}$ benutzen und genau dann eine Kante zwischen zwei Knoten einfügen, falls diese in unserer Relation $\mathcal{R}_{S_i}$ liegen.

\begin{definition}
	Der Graph $G_{S_i}$ einer Sequenz $S_i$ ist definiert als Graph über der Knotenmenge $V_{S_i}$ und den Kanten $E_{S_i}$. Es gilt
	\begin{equation}
		(u,v) \in E_{S_i} \Longleftrightarrow u,v \in V_{S_i}, (u,v) \in E^{+}\: \text{und}\: u \preceq_{S_i} v \Longleftrightarrow u\: \mathcal{R}_{S_i}\: v
	\end{equation}
\end{definition}

Pfade von $v_{start}$ nach $v_{end}$ in $G_{S_i}$ sind genau die Teilmengen von \emph{partiellen Zuweisungsspalten}, die bezüglich $S_i$ \emph{konsistent} sind. Das liegt daran, dass für zwei Knoten $u,v \in V_{S_i}$ auf einem solchen Pfad sowohl $u \preceq_{S_i} v$, als auch $u \preceq_{S_i}^{*} v$ gelten und damit $u \preceq_{S_i}^{*} v \implies u \preceq_{S_i} v$. Das war aber genau die Definition für \emph{Konsistenz}: dass die Relation die natürliche Ordnung auf den Sequenzen erhält. Wie entfernen daher alle \emph{Stellen} unserer Sequenz aus den Knoten, die nicht auf dem gewählten Pfad liegen, weil diese die \emph{Konsistenz} verletzen würden. \unsure{Reicht das als Beweis oder muss das mit dem DAG aus dem Lemma gezeigt werden?}

Wenn wir jetzt für jede unserer Sequenzen $S_j$ einen Pfad von $v_{start}$ nach $v_{end}$ in $G_{S_j}$ wählen und alle nicht-besuchten \emph{Stellen}  aus ihren Knoten entfernen, dann hält die \emph{Konsistenzbedingung} auf der Relation, die durch die übriggebliebenen Zusammenhangskomponenten induziert wird. Das Resultat ist also ein \emph{Alignment}. Da wir aber nicht irgendein \emph{Alignment} erhalten wollen, sondern ein möglichst großes, wählen wir für jede Sequenz $S_i \in S$ den Pfad maximaler Länge durch $G_{S_i}$. Auf diese Weise löschen wir die minimale Anzahl an \emph{Stellen} aus ihren Zusammenhangskomponenten. Formal: Sei für eine Sequenz $S_i$ $g_{S_i}$ ein Pfad maximaler Länge $(v_{start}, u_1, \dots, u_n, v_{end})$ gegeben. Wir iterieren über alle Knoten $C \in \mathcal{C}_{S_i}$ und entfernen \emph{Stellen} $(i,p) \in C$, falls $C \notin g_{S_i}$. Wenn wir das für alle Sequenzen tun, dann nennen wir die Menge der reduzierten Zusammenhangskomponenten $C°$. Weil die mit $C°$ korrespondierende Relation ein \emph{Alignment} ist, wäre der \emph{Sukzessionsgraph} $SG(C°)$ ein DAG. Da die Pfade für jede Sequenz unabhängig voneinander sind, können wir diese in beliebiger Reihenfolge wählen, ohne dass dies einen Einfluss auf unser Ergebnis hat.

Letztendlich folgt, dass das Problem die \emph{partiellen Zuweisungsspalten} auf \emph{konsistente Zuweisungsspalten} zu reduzieren auf die Suche nach Pfaden maximaler Länge durch gerichtete azyklische Graphen abbildbar ist \citep{cpm10}.

\subsubsection{Beispiel zum Entfernen von Stellen mit dem Algorithmus von Pitschi}

Bevor wir im nächsten Abschnitt genauer kennenlernen, wie man einen längsten Pfad im DAG eigentlich genau berechnet, betrachten wir unser Beispiel, in diesem Fall nur für die Sequenz $S_1$. Weil der Algorithmus zum Berechnen des längsten Pfades die Kantengewichte ignoriert und nur Knoten zählt, wurden diese nicht mehr angegeben. Formal handelt es sich hierbei um die Suche nach einem längsten Pfad in einem ungewichteten Graphen. Die Pfade, die über die Bildung der transitiven Hülle neu dazu kamen, sind grau angedeutet.

\begin{center}
	\begin{tikzpicture}[
	mycircle/.style={
		circle,
		draw=black,
		fill opacity = 0.3,
		text opacity=1,
		inner sep=0pt,
		minimum size=15pt,
		font=\tiny},
	myarrow/.style={-Stealth},
	node distance=0.6cm and 1.1cm
	]
	
	\node[mycircle,fill=gray] at (0.5,-0.5) (c1) {$v_{start}$};
	\node[mycircle,fill=magenta] at (2,2) (c2) {\begin{tabular}{c}
		$(1,2)$ \\ $(4,1)$
		\end{tabular}};
	\node[mycircle,fill=lime] at (4,2) (c3) {\begin{tabular}{c}
		$(1,3)$ \\ $(2,1)$ \\ $(4,2)$
		\end{tabular}};
	\node[mycircle,fill=blue] at (4,-0.5) (c4) {\begin{tabular}{c}
		$(1,4)$ \\ $(2,2)$ \\ $(3,3)$ \\ $(4,3)$
		\end{tabular}};
	\node[mycircle,fill=green] at (6,-0.5) (c5) {\begin{tabular}{c}
		$(1,5)$ \\ $(2,3)$ \\ $(3,4)$ \\ $(4,4)$
		\end{tabular}};
	\node[mycircle,fill=red] at (8,-0.5) (c6) {\begin{tabular}{c}
		$(1,8)$ \\ $(2,4)$ \\ $(3,5)$ \\ $(4,5)$
		\end{tabular}};
	\node[mycircle,fill=blue] at (8,2) (c8) {\begin{tabular}{c}
		$(1,6)$ \\ $(2,7)$ \\ $(4,8)$
		\end{tabular}};
	\node[mycircle,fill=green] at (10,2) (c9) {\begin{tabular}{c}
		$(1,7)$ \\ $(2,8)$ \\ $(4,9)$
		\end{tabular}};
	\node[mycircle, fill=gray] at (12,-0.5) (c10) {$v_{end}$};
	
	\foreach \i/\j/\txt/\p in {% start node/end node/text/position
		c1/c8//above,
		c2.310/c10//above,
		c3/c10//above,
		c4/c8//above,
		c4/c9//above,
		c5/c9//above,
		c8/c10//above}	
	\draw [myarrow,lightgray] (\i) -- node[font=\tiny,\p] {\txt} (\j);
	
	\foreach \i/\j/\txt/\p in {% start node/end node/text/position
		c1/c5//above,
		c1/c6//above,
		c1/c10//above,
		c4/c5//above,
		c4/c10//above,
		c5/c10//above}	
	\draw [myarrow,lightgray] (\i) to[bend right] node[font=\tiny,\p] {\txt} (\j);
	
	\foreach \i/\j/\txt/\p in {% start node/end node/text/position
		c2/c8//above,
		c2/c9//above,
		c3/c9//above}	
	\draw [myarrow,lightgray] (\i) to[bend left] node[font=\tiny,\p] {\txt} (\j);
	
	\foreach \i/\j/\txt/\p in {% start node/end node/text/position
		c1/c3//below,	
		c1/c4//above,
		c5/c6//above,
		c8/c9//above,
		c6/c10//above}	
	\draw [myarrow] (\i) -- node[font=\tiny,\p] {\txt} (\j);
	
	\foreach \i/\j/\txt/\p in {% start node/end node/text/position
		c1/c2//above,
		c2/c3//above,	
		c3/c4//right,
		c4/c5//above,
		c5/c8//above,
		c8/c9//above,
		c9/c10//above}	
	\draw [myarrow,red] (\i) -- node[font=\tiny,\p] {\txt} (\j);
	
	\end{tikzpicture}
\end{center}

\vspace{-5pt}

Im Graphen $G_{S_1}$ gibt es eine Kante vom $(1,5)$er-Knoten zum $(1,6)$er, obwohl diese direkte Kante in $G^{*}$ gelöscht wurde. Das liegt an der Verbindung $(2,3) \rightarrow (2,4) \rightarrow (2,5) \rightarrow (2,7)$ und $(1,5) \preceq_{S_1} (1,6)$. Die Kante von $(1,8) \rightarrow (1,6)$ gibt es aber nicht, obwohl diese in der transitiven Hülle verbunden sind. Das liegt an $(1,8) \npreceq_{S_i} (1,6)$. Der hellblaue Knoten enthält keine \emph{Stelle} aus $S_1$ und ist daher nicht Teil von $V_{S_1}$.

Wie schon beim Blick auf den ersten Sukzessionsgraphen vermutet, ist das Problem die \emph{Stelle} $(1,8)$, die für Überkreuzungen sorgt. Sie liegt nicht auf dem längsten Pfad (rot markiert) und wird daher aus der Zusammenhangskomponente des Knoten entfernt. Die längsten Pfade aller anderen Sequenzen besuchen auch alle Knoten der jeweiligen Graphen, weshalb keine weiteren Knoten modifiziert werden müssen.

\subsection{Algorithmus zur Bestimmung des längsten Pfads}

Für die Berechnung der längsten Pfade benutzen wir die \emph{topologische Sortierung} der Knoten der Graphen $G_{S_i}$ für alle Sequenzen $S_i \in S$. 

\begin{definition}[Topologische Sortierung]
	Es sei ein Graph $G = (V,E)$ gegeben. Eine \emph{topologische Sortierung} von $G$ ist eine lineare Aufzählung der Knoten in $V$ $f(v_1), \dots, f(x), \dots, f(y), \dots, f(v_n)$ bei der für alle Knoten $x,y \in V$ gilt:
	$f(x) < f(y) \implies $  
\end{definition}

Algorithmus für TS

Beweisen, dass Algo TS liefert auf DAG

Algorithmus für längsten Pfad

Beweis, dass Algo längsten Pfad liefert

Beispiel an einem Graphen

\subsection{Probleme bei der Heuristik zum Entfernen von Kanten}

circular permutation -> alle Kanten gelöscht

viele Kanten entfernt, die nichts mit dem Zyklus zu tun haben

\subsection{Verankerungen}

\cite{mpps06} haben einen semiautomatisierten Ansatz für das \emph{multiple sequence alignment}-Problem entwickelt. Bei diesem kann der Benutzer, der über Expertenwissen verfügt, vorgeben, welche Positionen der Sequenzen auf jeden Fall miteinander \emph{aligniert} werden sollen. Das ist insbesondere dann nützlich, wenn es Abschnitte gibt, die sich mathematisch gesehen sehr ähnlich sind, deren Zuordnung biologisch aber falsch wäre. Ein Beispiel für solche Sequenzen haben wir mit den \emph{Hoxgenen} des Pufferfisches im Abschnitt über die Schwächen von DIALIGN bereits kennengelernt. Es hat sich herausgestellt, dass das semiautomatisierte Verfahren bei vielen Tests auf der Referenzdatenbank BAliBASE bessere Ergebnisse geliefert hat, wenn die Startpunkte aller Motive als \emph{Verankerungen} gesetzt wurden.

Das Verfahren funktioniert so, dass der Experte \emph{Fragmente} (die auch nur paarweise Zuweisungen sein können) wählt und diese mit einer Dringlichkeit an DIALIGN übergibt. Wie im Standardalgorithmus werden dann gierig die \emph{Fragmente} mit maximaler Dringlichkeit gewählt, die zu allen bereits gewählten \emph{konsistent} sind. Zwischen diesen \emph{Verankerungen} wird dann ganz normal DIALIGN durchgeführt, wodurch die restlichen Abschnitte automatisiert \emph{aligniert} werden.

\cite{cpm10} benutzten die bereits vorliegende Infrastruktur in DIALIGN, um die \emph{Zuweisungsspalten}, die der Algorithmus von Pitschi geliefert hat, als \emph{Verankerungen} ins \emph{Alignment} zu integrieren. Weil diese Möglichkeit in unserer Implementierung nicht bereits vorhanden ist, werden wir die Resultate des \emph{Min-Cut}-Ansatzes direkt mit Hilfe von \textrm{EdgeAddition} und dem \emph{Alignmentgraphen} in unser Ergebnis integrieren. Neben dem Übergeben von \emph{konsistenten Zuweisungsspalten} als \emph{Verankerungen} hat man auch Versuche mit \emph{partiellen Zuweisungsspalten} gemacht. Hier war die gierige Heuristik von DIALIGN aber nicht erfolgreich und die Ergebnisse waren schlechter als mit dem Algorithmus von Pitschi. Aus diesem Grund werden wir den zweiten Ansatz nicht weiter behandeln.

\subsection{Komplexität}

$\oh(n^3\cdot L)$ für n-faches Topological Sort auf einem Knoten mit $\oh(n^2\cdot L)$-vielen Kanten im schlimmsten Fall.

\section{Abschluss und Zusammenfassung}

\subsection{Gesamtkomplexität}
$\oh(n^4\cdot L^{7/2})$ 

\subsection{Evaluierung}
