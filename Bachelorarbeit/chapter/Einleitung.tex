\chapter{Einleitung und Motivation}
\label{ch:einleitung}

Das Berechnen von multiplen Sequenzalignments ist eine fundamentale Technik mit einer Vielzahl von Anwendungsgebieten in der Biologie, aber auch in anderen Disziplinen wie dem \emph{Natural Language Processing}. Ziel von Sequenzalignments ist es, für eine Menge von Zeichenketten aus einem endlichen Alphabet Zuordnungen zwischen den einzelnen Symbolen zu finden, sodass möglichst ähnlich Symbole oder ganze Abschnitte einander zugeordnet sind. Man versucht auf diese Weise funktionelle, strukturelle oder evolutionäre Ähnlichkeiten aufzuspüren. Ein Beispiel für relevante strukturelle Ähnlichkeiten sind Proteinsequenzen. Wenn Proteine aus ähnlichen Aminosäuren in vergleichbaren Reihenfolgen aufgebaut sind, dann kann man davon ausgehen, dass diese auch eine ähnliche 3D-Struktur und ähnliche Funktionen haben, selbst wenn sie in unterschiedlichen Organismen vorkommen. Im Laufe der Evolution verändern sich aufgrund von Mutationen DNA und Proteine von Arten. Diese Vorgänge sind die Ursache für den beispiellosen Reichtum an Lebewesen auf der Erde. Mit Hilfe von Sequenzalignments kann man nachvollziehen, wie diese Entwicklung vonstatten gegangen ist. Zu den häufigsten Mutationsarten bei Genmutationen (im Gegensatz zu Genommutationen und Chromosomenmutationen) gehören Punktmutationen, bei denen eine einzelne DNA-Base durch eine andere ersetzt wird, sowie Deletionen und Insertionen, bei denen ganze Abschnitte einer Sequenz gelöscht oder eingefügt wurden. Glücklicherweise besitzen unsere Alignments Möglichkeiten, genau diese Situationen abzubilden. Informell könnte man sagen, dass man bei einer Zuweisung die Elemente der Eingabestrings einander so zuordnet, dass jedem Symbol genau ein Symbol jeder anderen Sequenz oder eine neu eingefügte Lücke, \emph{Gap} genannt, zugeordnet ist. Dabei darf die Reihenfolge der Elemente nicht verändert werden.

Betrachten wir dazu zwei kleine Beispiele von Wörtern, die häufig falsch geschrieben werden:

\ttfamily
\begin{center}
\begin{tabular}{|ccc|}
	\hline
		OR-GINAL & \hspace{2cm} & SYLVESTER \\
		ORIGINAL & \hspace{2cm} & SILVESTER \\
	\hline
\end{tabular}
\end{center}
\normalfont

Im ersten Fall wurde bei der falschen Schreibweise ein benötigter Buchstabe weggelassen. Damit es trotzdem zu einer passenden Zuordnung der anderen Buchstaben kommt, wurde in die erste Sequenz eine Lücke (-) eingefügt. Im evolutionären Kontext wäre dies ein Beispiel für eine Deletion. Beim zweiten Wortpaar wurde ein Buchstabe durch einen anderen, fehlerhaften ersetzt. Das ist ein klassisches Beispiel für eine Punktmutation oder einen Einzelnukleotid-Polymorphismus. Die Berechnung eines Sequenzalignments ist in vielen Fällen der erste Schritt einer Sequenzanalyse in der Molekularbiologie \cite{cpm10}. Diese Analysen dienen unter anderem der Bestimmung, ob Sequenzen miteinander verwandt sind (Homologie), von Markergenen oder der direkten Schlussfolgerung von Sequenzen auf ihre molekulare Struktur. 

Ein zweites großes Einsatzgebiet des \emph{Multiple-Sequence-Alignment}-Problems ist das \emph{Natural Language Processing}, also die maschinelle Verarbeitung menschlicher Sprache \cite{s10}. Sätze, Wörter oder Ausdrücke können aligniert werden, um mechanisch Sätze zu übersetzen oder Texte zusammenzufassen. Noch einen Schritt weiter geht das Alignieren von \emph{Phonemen}, unter anderem von der textuellen Darstellung auf die Aussprache geschlossen oder umgekehrt Sprache textuell darzustellen. Obwohl viele dieser Anwendungen nur auf zwei Sequenzen arbeiten, wie beispielsweise einem Text und seiner Übersetzung, gibt es auch Fälle bei denen multiple Alignments nötig sind. Dazu gehören unter anderem Vergleiche von Texten, die in anderen Worten denselben Inhalt wiedergeben, oder von gleichbedeutenden Worten aus unterschiedlichen Sprachen der selben Sprachfamilie. In der Sprachentwicklung gibt es interessante Parallelen zu den evolutionären Vorgängen in Genomen. Im Kontext dieser Bachelorarbeit werden wir uns jedoch auf Anwendungen in der molekularen Bioinformatik beschränken.  

\section{Ablauf der Bachelorarbeit}

Im zweiten Kapitel führen wir mit dem Algorithmus von Needleman-Wunsch einen Standardansatz für paarweise Alignments ein. Am Beispiel dieses Algorithmus beschäftigen wir uns mit der Komplexität des \emph{Multiple-Sequence-Alignment}-Problems und stellen fest, dass es für Zuweisungen zwischen mehr als nur einigen wenigen Sequenzen nicht zielführend ist, diese mathematisch exakt zu berechnen. Um diesem Problem Herr zu werden, lernen wir im Laufe der Bachelorarbeit zwei ausgefeilte Heuristiken für multiple Sequenzalignments ein. Zunächst lernen wir mit DIALIGN einen Algorithmus kennen, der im Gegensatz zu Needleman-Wunsch nicht auf der Basis von einzelnen Symbolen Alignments konstruiert \cite{mdw96}. Stattdessen werden ganze Segmente als Bausteine der Zuweisungen benutzt. Der zweite Algorithmus ist der graphtheoretische Ansatz von Corel et al.\cite{cpm10}. Dieser basiert zwar auch auf DIALIGN, hat aber den Anspruch in Situationen, bei denen die Heuristik von DIALIGN falsche Entscheidungen trifft, bessere Ergebnisse zu liefern. Exemplarisch wird danach ein wichtiger Schritt des Verfahrens als Computerprogramm umgesetzt. Dabei wird mit der Programmiersprache C++ und der \emph{Boost Graph Library} gearbeitet.








